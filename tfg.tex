\documentclass{tfg_domingo}
% \documentclass[numeros]{tfg_domingo}
\usepackage{amsmath}
\usepackage{amssymb}
\usepackage{bm}
\usepackage{bbm}

\autor{Pedro Castro Cutillas}
\titulo{no lo se}
% Título corto para los encabezamientos de pagina:
\corto{} % En blanco si no es necesario recortarlo.
\ingles{Efficient sampling for different scales.\\
Stereology applied to AI.}
\fecha{Julio - 2024}
% La normativa prescribe «cuatro o cinco palabras clave, en
% español y en inglés, para su indexación en el repositorio
% de TFG».
\palabras{Estereología, IA, Conteo de personas, Eficiencia}%
  {Stereology, AI, Crowd Counting, Efficiency}

\usepackage{amsthm} % Esto solo es relleno.
\usepackage{lipsum}
\usepackage{graphicx}
\usepackage{caption}
\usepackage{subcaption}
\usepackage{comment} 


\newtheorem{Def}{Definition}
%\newtheorem{Proof}{Proof}
\newtheorem{Th}{Theorem}
\newtheorem{Algorithm}{Algorithm}
\newtheorem{Lemma}{Lemma}
\newtheorem{Prop}{Proposition}
\newtheorem{Corollary}{Corollary}
\providecommand{\abs}[1]{\lvert#1\rvert}
\providecommand{\norm}[1]{\lVert#1\rVert}
\renewcommand{\exp}[1]{e^{#1}}

\begin{document}

% Si alguna palabra se divide entre dos líneas en un punto
% indebido, podemos indicar aquí los puntos de corte
% aceptables (si los hay), p. ej,
% \hyphenation{ba-rro-co, frío, cria-do, su-per-ra-tón}
\hyphenation{Dijkstra new-speak}

\portada
\frontmatter
% \sucinto{A Sofía}
\gracias{\input{agradecimientos.txt}}
\resumen{%El problema del conteo de personas en imágenes resulta de vital importancia en tareas de seguridad pública y control de masas en eventos multitudinarios. La tecnología actual utiliza diferentes metodologías automáticas para diseñar y entrenar distintos modelos de redes neuronales. Por otro lado, existen métodos de conteo manual donde el objetivo es obtener una estimación con el menor número de muestras posibles. Los métodos manuales se basan en una ciencia matemática llamada estereología, pero su aplicación ha quedado en segundo plano por la ingente cantidad de datos que ahora permiten entrenar redes neuronales con bastante precisión. En este trabajo se plantea hacer una revisión sistemática de la literatura de los dos campos. Se tratará de combinar los métodos de estereología con las redes neuronales para comprobar si la estereología ha quedado suplantada o sus fundamentos pueden lograr avances en este campo. 



%La estereología es la ciencia del muestreo geométrico y permite estimar cantidades utilizando fórmulas que suelen incluir esperanzas. En este trabajo se han intentado mejorar los estimadores usados en estereología aproximando dichas esperanzas mediante el uso de integración QMC con $N$ puntos óptimos que minimizan el \textit{worst-case-error} para el \textit{reproducing kernel Hilber space} $H_{mix}^1$. Este proceso debería dar estimadores con poca varianza, mejorando la calidad de la cantidad estimada.\\
%Para comprobar el funcionamiento de este proceso se han seleccionado tres \textit{test systems} conocidos, el \textit{test system de quadrats}, el \textit{Buffon test system} y el \textit{Buffon-Steinhaus test system}, y se ha implementado en un código python para cada uno.\\
%Los resultados calculados por ordenador apuntan a que este proceso puede que funcione mejor o peor dependiendo de la isotropía, anisotropía o simetría que tiene el objeto de cuya cantidad se quiere estimar. %También es posible que cambiando un cierto parámetro $\gamma$ (ver \cite{Hinrichs.pdf}) se obtengan mejores resultados al usar ese proceso, pero aún así permite obtener estimadores precisos con bajo error.\\

%----------------------------------------------------------------------------------
%Stereology is the science of geometric sampling and it allows to estimate quantities using formulas that usually include expectations. Artificial Intelligence is a discipline in computer science that encompasses several different processes concerned with making machines smart. In this work we combined both Stereology and AI in order to test a new method for Crowd Counting, an AI field that revolves around the estimation of the number of individuals in images or videos. Our method's approach considers a modified test system of quadrats whose quadrats are only located at the points contained in an Optimal Point Set that minimizes the worst case error for Quasi-Monte Carlo Integration. By superimposing the system to an image and cropping that image by the quadrats, we can then use an AI model to count individuals on each crop, subsequently using Equations \ref{Conteo_usada} and \ref{QMC} to estimate the total count value of individuals in the original image. The chosen AI model to try our method was CLIP-EBC, as it is a public model and one of the best performing models in crowd-counting. The computed results for the validation metrics MAE and RMSE suggest that there is a lot or research to be made regarding our method, however, the best variant tested for our method is not far from performing in a similar level to that of the state-of-the-art methods in Crowd Counting. What's more, the computing times obtained for that variant are much lower than those obtained by one of the best performing methods, CLIP-EBC.\\

La estereología es la ciencia del muestreo geométrico y permite estimar cantidades utilizando fórmulas que suelen incluir esperanzas. La inteligencia artificial es una disciplina en ciencias de computación que engloba varios procesos relacionados con crear máquinas inteligentes. En este trabajo se han combinado la estereología y la IA de cara a probar un nuevo método de conteo de multitudes (Crowd Counting), el área centrada en estimar el número de individuos en imágenes o vídeos. El planteamiento de nuestro método considera un \textit{test system} (sistema de testeo) de quadrats modificado cuyos quadrats sólo se posicionan en los puntos contenidos en un conjunto de puntos óptimo que minimiza el \textit{worst case error} (error en el peor caso) para la integración Quasi-Monte Carlo. Superponiendo este \textit{test system} a una imagen y recortándola en los quadrats podemos usar el modelo de IA para contar individuos en cada recorte y, posteriormente, usar las Ecuaciones \ref{Conteo_usada} y \ref{QMC} para estimar la cantidad total de individuos en la imagen original. El modelo de IA escogido para probar nuestro método ha sido CLIP-EBC, dado que es un modelo público y uno de los modelos con mejores resultados en conteo de multitudes. Los resultados computados para las métricas de validación MAE (error absoluto medio) y RMSE (error cuadrático medio) muestran resultados esperanzadores respecto a nuestro método. %%DomingoMaster: Esto, aunque esta bien, me parece un poco repetitivo.
%Aunque, la mejor variante probada de nuestro método no está lejos de rendir a un nivel similar al de los métodos más recientes de conteo de multitudes. 
Lo que es más, los tiempos de cómputo obtenidos para dicha variante son mucho más bajos que los obtenidos por uno de los modelos que mejores resultados obtiene, CLIP-EBC.\\





}{%Stereology is the science of geometric sampling and it allows to estimate quantities using formulas that usually include expectations. In this work we tried to improve stereology estimators by approximating the expectations using QMC integration with $N$ optimal points that minimize the worst case error in the reproducing kernel Hilbert space $H_{mix}^1$. This process should provide estimators with low variances, improving the quality of the estimated quantity.\\
%In order to test this process we selected three known test systems, namely the test system of quadrats, the Buffon test system and the Buffon-Steinhaus test system, and implemented it in python code for each one of them.\\
%The computed results led us to think that this process might perform differently depending on the isotropy, anisotropy or symmetry that the object whose quantity we estimate has. %Also, it's possible that changing a certain parameter $\gamma$ (see \cite{Hinrichs.pdf}) may provide better results when using that process, but it still gives low error accurate estimators.\\
(tiempo presente, primera persona del plural, hablando de lo que se ha conseguido en el TFG)
Stereology is the science of geometric sampling and it allows to estimate quantities using formulas that usually include expectations. Artificial Intelligence is a discipline in computer science that encompasses several different processes concerned with making machines smart. In this work we combined both Stereology and AI in order to test a new method for Crowd Counting, a field that revolves around the estimation of the number of individuals in images or videos. Our method's approach considers a modified test system of quadrats whose quadrats are only located at the points contained in an Optimal Point Set that minimizes the worst case error for Quasi-Monte Carlo Integration. By superimposing the system to an image and cropping that image by the quadrats, we can then use an AI model to count individuals on each crop, subsequently using Equations \ref{Conteo_usada} and \ref{QMC} to estimate the total count value of individuals in the original image. The chosen AI model to test our method was CLIP-EBC, as it is a public model and one of the best performing models in crowd-counting. The computed results for the validation metrics MAE and RMSE shows promising results.
%%DomingoMaster: Creo que da un poco repetitivo.
%however, the best variant tested for our method is not far from performing in a similar level to that of the state-of-the-art methods in Crowd Counting. 
An additional benefit is  the computing times obtained for that variant are significantly lower than those obtained by the most optimised performing methods, CLIP-EBC.\\

}
\tableofcontents

\mainmatter
\chapter{Introduction}\label{cap:Intro}
\section{Motivación}
Los compiladores son un pilar fundamental en la tecnología contemporánea. Estos han conseguido facilitar la programación construyendo capas de abstracción entre la máquina y el programador. Antes se programaba directamente en código ensamblador que después se traducía al código máquina para generar un ejecutable. A día de hoy, lo más común es programar en lenguajes de alto nivel como \textit{Python} o \textit{JavaScript}, donde hay miles de librerías y herramientas que ofrecen múltiples funcionalidades para que facilitar el desarrollo de software de cualquier tipo. \\
Se debe hacer una clara distinción entre compilador e interprete, el compilador traduce todo el código fuente del lenguaje original a código máquina, lo cual evidentemente le hace en la mayoría de casos más rápido. Sin embargo, el interprete convierte cada línea, una a una a código máquina teniendo en cuenta los resultados de líneas anteriores con técnicas como el conteo de referencias, garbage collection, etc... \\
En la primera parte de este trabajo hablaremos del desarrollo de un compilador. He escogido escribir un compilador porque siempre me ha interesado conocer lo que realmente ocurre en la computadora cuando tras escribir tu programa genera un ejecutable. Y porque creo que conocer detalles de programación de bajo nivel son de ayuda en el día a día de un programador. Además, conocer todas las etapas por las que pasa el código fuente hasta convertirse en código máquina te da una perspectiva global sobre la naturaleza de un lenguaje de programación de alto nivel.\\
En cuanto a la segunda parte del trabajo, los algoritmos de compresión siempre habían llamado mi atención. Sin embargo, nunca me había tomado el tiempo para analizar ninguno de ellos. Con este trabajo he tenido la oportunidad de profundizar en conceptos como los registros de desplazamiento, o las formulas lógicas que antes desconocía.\\
En conclusión, la motivación principal de este trabajo era conocer en mayor profundidad los compiladores y los algoritmos de compresión, que en cierta manera se entremezcla con algunos algoritmos de cifrado.

\section{Objetivos}
El primer objetivo de este proyecto era crear un compilador funcional del lenguaje BeGone, siguiendo las lecciones del curso de David Beazley. El curso contiene lecciones básicas para guiarte, algunos tests, y un compilador medio implementado para que compruebes si la salida de tu compilador coincide con la de este. Este compilador de ayuda deja de ser útil a partir de la implementación básica del módulo de generación de código llvm. A partir de ese punto, hay algunos tests, pero en general te las tienes que apañar solo. Por lo que la implementación es cien por cien escrita por el autor de este trabajo.\\\\
Después del desarrollo completo del compilador, mi tutor me ofreció un artículo \cite{limniotis2007nonlinear} donde había un algoritmo que daba salida a un polinomio basándose en una secuencia binaria. Y el problema planteado fue si ese polinomio era reducible, eso derivo a desarrollar un formato de archivo especial para este algoritmo. Y con este formato, implemente en el compilador original dos funciones nativas para codificar y decodificar un archivo. Evidentemente, esto provoco cambios en todas las etapas del compilador. Al hacer esto el compilador original paso de llamarse BeGone a GoneFSR. \\\\
El último de los objetivos fue desarrollar una memoria técnica donde todo fuera claro para el lector, para ello, tome como estructura de la sección del desarrollo del compilador, el documento \href{https://repositorio.unican.es/xmlui/handle/10902/30046}{intérprete Lox} ofrecido por mi tutor. Para la segunda parte del proyecto, los conceptos se explican desde los más sencillos hasta los más complejos, para que el lector no reciba estos últimos de golpe.

\section{Desarrollo software}
Para el desarrollo del compilador el modelo de desarrollo software utilizado ha sido el modelo incremental iterativo, tal y como se recomienda en el curso de BeGone. Este consiste en dividir el proyecto en etapas, y para cada etapa aplicar las distintas fases de desarrollo para producir un software sin errores. Estas etapas se desarrollan secuencialmente, aplicando pequeños incrementos en la complejidad del proyecto después de haberse asegurado que la etapa anterior estaba bien desarrollada. En cada incremento el objetivo es añadir o extender una funcionalidad en el proyecto global, a cada incremento se lo podría asociar con un hito si se conoce el \textit{framework} \textit{SCRUM}. \\\\
Al desarrollar el proyecto de esta manera puedes aplicar tests en cada etapa, haciendo así que el feedback dado por los tests tanto unitarios (caja negra) como de sistema (caja blanca) sea más fácil de entender y aplicar las correcciones necesarias implique menos trabajo. \\\\
En la práctica he creado un repositorio git en local y he trabajado sobre dos ramas una \textit{dev} y otra \textit{prod}. Haciendo commit en \textit{dev} por cada pequeño avance, y haciendo merge en \textit{prod} cada vez que completaba una etapa del desarrollo. Podéis consultar el código en \href{https://github.com/domingoUnican/TFGPedroCastro/tree/main/code}{repositorio}. \\\\
En el repositorio referenciado anteriormente no encontrareis el historial de logs pertinente. Ya que el compilador fue desarrollado en Mayo, y no subí el repositorio original a mi github. \\\\
Por último, respecto al desarrollo de \textit{scripts} y otros experimentos relacionados con la segunda parte del proyecto, no he seguido ningún estándar. Ya que solían ser pruebas separadas y no un proyecto global, que necesitase de planificación y estructura.


\chapter{Underlying theory of our method}\label{cap:Teoria}
%%%Estructura:
%
%[No teoría de probabilidad (TFG) porque vamos a usar MAE???
% Quizás puedo meterlo porque la varianza del método es básicamente un error
% tal que el resultado final tenga un error de +- algo y luego puedo ver
% si la cantidad de personas real está dentro del intervalo de error del método???
% Si meto teoría de probabilidad sería: 
%%Introducción a media y varianza probabilisticas
%%Adaptación de la varianza al test system que usamos]
%
%[Ahora no hay QuasiMonteCarlo Integration, lo que hay es:]
%%Teoría sobre IA/Machine learning, funcionamiento de capas, cómo funciona
% el modelo que usamos (GauNet o FGENet), funciones de pérdida y de error (MAE), etc. (OBS que la varianza es del valor real de personas y la función de pérdida es la que se usa mientras se entrena el modelo, la varianza va ligada al resultado de toda la imagen, la función de perdida va ligada al resultado de entrenar con cada recorte) + problemas de la IA en reconocimiento de personas

%%Cómo vamos a integrar IA + Estereología, explicar point-sets y funcionamiento de 
% lo que hará el programa.
%%%%%%%%%%%%%%%%%%%%%%%%%%%%%%%%%%%%%%%%%%%%%%%%%%%%%%%%%%%%%%%%%%%%%%%%%%%%%%%%%%%%%%%%%%%%%%%%%%%%%%%%%%%%%%%%%%%%%%%%%%%%%%%%%%%%%%%%%%%%%%%%%%%%%%%%%%%%%%%%%%%%%%%%%%%%%%%%%%%%%%%%%%%%%%%%%%%%%%%%%%%%%%%%%%%%%%%%%%%%%%%%%%%%%%%%%%%%%%%%%%%%%%%%%%%%%%%%%%
%%%%%%%%%%%%%%%%%%%%%%%%%%%%%%%%%%%%%%%%%%%%%%%%%%%%%%%%%%%%%%%%%%%%%%%%%%%%%%%%%%%%%%
%%Teoría de probabilidad
\section{Probability Theory}
This section introduces the relationship between probability theory
and Hilbert spaces. We will see that better estimators relate to sampling points that minimize some quantity. The section aims to connect all these subject in a rigorous way.
%Intro a media y varianza
\subsection{Mean and Variance}
%%%%%%%%Media y varianza 
%(Empezar con second order stereology en CO.IAS.17.Hist.pdf) y 
% seguir con formulas generales??? 
% y formulas para cada problema.
% Domingo: Mejor previous chapter
As it was shown in the previous chapter, estimation problems in Stereology generally make use of the expectation (namely the mean) in order to get a better estimation of the desired quantity. If one wants to know how good of an estimation was obtained, the variance comes into play. However, since the real quantity that we want to estimate is unknown, 
it is necessary to predict the error as well with \textit{variance predictors}, which are usually employed to fulfill the same role as the usual variance.\\

%Domingo: Some reformating, also we did not introduce what is systematic sampling.
We follow the naming convention in Stereology, stating that the mean of a random variable is called a \textit{first order property} and the variance is a \textit{second order property}. We remark that formulas for the variance of %when mean of 
independent estimations are known. However, under systematic sampling the sampled items are correlated to unknown degrees depending on the population pattern, thus the problem is non trivial unless the
%Domingo: Hay que definir que es una random permutation
population is a random permutation (\cite{CO.IAS.17.Hist.pdf}).\\

In probability, we have the following definitions for the mean and the variance of real valued random variables.\\

%Domingo: He añadido la definicion de varianza aqui
\begin{Def}
    Given a sample space $\Omega$, an event space $\sigma$ and a probability function $\mathbb{P}$. If $X$ is a discrete, real valued random variable defined in a probability space $(\Omega, \sigma, \mathbb{P})$, the \textit{mathematical expectation of $X$} and the \textit{variance of $X$} are defined as
    \begin{equation*}
        \mathbb{E}[X]=\sum_n x_n \cdot \mathbb{P}_X[x_n] = \sum_n x_n \cdot \mathbb{P}[X=x_n], \quad Var(X)=\mathbb{E}[(X-\mathbb{E}[X])^2]
    \end{equation*}
    given that the sums exist.
\end{Def}
\vspace{2mm}
\begin{Def}
    Given a sample space $\Omega$, an event space $\sigma$ and a probability function $\mathbb{P}$. If $X$ is a real valued random variable defined in a probability space $(\Omega, \sigma, \mathbb{P})$ such that it's density function $f_X$ exists, the \textit{mathematical expectation of $X$} can be defined as
    \begin{equation*}
        \mathbb{E}[X]=\int_{-\infty}^{\infty} t \cdot f_X(t)\,\mathrm{d}t,
    \end{equation*}
    given that this expression makes sense.
\end{Def}

\vspace{2mm}
Furthermore, a very useful proposition says as follows.\\

\begin{Prop}
    Given a sample space $\Omega$, an event space $\sigma$  and a probability function $\mathbb{P}$. If $X$ is a real valued random variable defined in a probability space $(\Omega,\sigma,\mathbb{P})$ such that $\mathbb{E}[\abs{X}^2]<\infty$, then
    \begin{equation*}
        Var(X)=\mathbb{E}[X^2]-(\mathbb{E}[X])^2
    \end{equation*}
\end{Prop}
\begin{proof} By definition, $Var(X)=\mathbb{E}[(X-\mathbb{E}[X])^2]$. What's more, $$ (X-\mathbb{E}[X])^2 = X^2 + (\mathbb{E}[X])^2 - 2X\mathbb{E}[X]. $$
Given that $\mathbb{E}[X]$ is a real value, applying the expectation we have
\begin{multline*}
    \mathbb{E}[(X-\mathbb{E}[X])^2] = \mathbb{E}[X^2] + \mathbb{E}[(\mathbb{E}[X]^2)] - \mathbb{E}[2X\mathbb{E}[X]] \\ = \mathbb{E}[X^2] + (\mathbb{E}[X])^2 - 2\mathbb{E}[X]\mathbb{E}[X] = \mathbb{E}[X^2] - (\mathbb{E}[X])^2.
\end{multline*}
This finishes the proof.
\end{proof}
\vspace{2mm}

% Mirar cuánto he copiado literal de Cavalieri_basics_R1.pdf 

%Reescribir con N en vez de T (hecho)
Regarding the error variance estimator, we will now show how to obtain one for the 1-dimensional case. Let $f$ be a predictor function such that it is periodic in $[0,1)$  %Domingo: Hay que cambiar que la funcion tenga soporte acotado por ser periodica.
and define the \textit{covariogram} as
\begin{equation} \label{eqCovariograma}
    g(h) := \int_0^1 f(x) \cdot f(x+h) \,\mathrm{d}x.
\end{equation}
This concept will be of utter importance, so first lets give some basic properties.\\

\begin{Prop}
    If $f$ is a periodic function in $[0,1)$ and $g$ is its covariogram, then the following properties apply:
    \begin{enumerate}
        \item $g$ is symmetric about $h=0$, namely: $g(h) = g(-h)$
        \item $\abs{g(h)} \leq g(0)$ (\textit{i.e.} $\sup_{h}{g(h)} = g(0)$)
        \item $\int_0^1 g(h) \,\mathrm{d}h = \Bigg\{ \int_0^1 f(x) \,\mathrm{d}x \Bigg\}^2 =: V^2$
    \end{enumerate}
\end{Prop}
\begin{proof} 
%Domingo: Asi se veia mal
We start by proving the first item
$$ g(-h) = \int_0^1 f(x) \cdot f(x-h) \,\mathrm{d}x =\{ y=x-h; dy=dx \}= \int_0^1 f(y+h) \cdot f(y) \,\mathrm{d}y = g(h).$$
For the second item
\begin{multline*}
        0 \leq \int_0^1 [ f(x+h) - f(x) ]^2 \,\mathrm{d}x = \int_0^1 f^2(x+h) \,\mathrm{d}x \hspace{1mm} + \int_0^1 f^2(x) \,\mathrm{d}x \hspace{1mm} - 2\cdot \int_0^1 f(x) \cdot f(x+h) \,\mathrm{d}x \\
        = g(0) + g(0) -2\cdot g(h)
    \end{multline*} 
For the last item, 
$$ \int_0^1 \,\mathrm{d}h \int_0^1 f(x) \cdot f(x+h) \,\mathrm{d}x = \int_0^1 f(x) \,\mathrm{d}x \int_0^1 f(x+h) \,\mathrm{d}h = \Bigg\{ \int_0^1 f(x) \,\mathrm{d}x \Bigg\}^2.$$
This finishes the proof.
%Domingo: Una cosa, las ecuaciones se puntuan con punto final.
\end{proof}

\vspace{2mm}

\textbf{Note:} We remark that if $f$ is integrable, then its covariogram is derivable. This will be important when considering the Hilbert space of functions defined later in this chapter.\\%Pregunta??????

%Domingo: Lo he escrito a toda prisa. Espero que este bien
We apply this to design an estimator for the following parameter of interest:
\begin{equation*}
    V = \int_0^1 f(x) \mathrm{d}x,
\end{equation*}
where $f$ is a periodic function, with period $1$.
Lets imagine that we take observations at $z+j/N$ with $j=0,1,2,\ldots, N-1$. An estimator for $V$ reads
\begin{equation*}
    \widehat{V} = \frac{1}{N} \cdot \sum_{j=0}^{N-1} f\left(z+\frac{j}{N}\right) \equiv v(z)
\end{equation*}
which is a periodic function of $z$ with period $1/N$, thus it suffices to consider $z\in (0,1/N)$. As a result, assuming $z \sim U(0,1/N)$ we have
\begin{equation*}
    \mathbb{E}[v(z)] = \int_0^{1/N} \frac{\,\mathrm{d}z}{1/N} \cdot \frac{1}{N}\cdot \sum_j f(z+j/N) = \sum_j \int_0^{1/N} f(z+j/N) \,\mathrm{d}z = \sum_j V_j = V,
\end{equation*}
that is, $v(z)$ is an UE of $V$.\\

Now, by definition,
\begin{equation*}
    Var(v(z)) = \frac{1}{1/N} \int_0^{1/N} v^2(z) \,\mathrm{d}z - V^2 = N \int_0^{1/N} v^2(z) \,\mathrm{d}z - V^2
\end{equation*}
Since $v(z)$ is periodic of period $1/N$, we can write it as a \textit{Fourier series},
\begin{equation*}
    v(z) = \sum_{k=-\infty}^{\infty} c_k \cdot \exp{2\pi i k zN},
\end{equation*}
where 
\begin{equation*}
    c_k = 
    \begin{cases}
         N \int_0^{1/N} v(z) \cdot \exp{(-2\pi i k z N)} \,\mathrm{d}z& (k>0)\\
         N \int_0^{1/N} v(z) \cdot \exp{(2\pi i k z N)} \,\mathrm{d}z& (k<0).
    \end{cases}
\end{equation*}
\vspace{2mm}
%Domingo: Tendriamos que mirar todos los sitios donde has utilizado \exp, no me gusta como queda. Quizás \renewcommand{\exp}[1]{e^{#1}}
Substituting $v(z)$, for $k>0$ we can write
\begin{equation*}
    c_k = N \int_0^{1/N} \frac{1}{N} \cdot \sum_{j=0}^{N-1} f(z+j/N) \cdot \exp{(-2\pi i k z N)} \, \mathrm{d}z = \int_{0}^{1} f(z) \cdot \exp{(-2\pi i k z N)} \,\mathrm{d}z
\end{equation*}
and the conjugate for $k<0$.\\

Using \textit{Parseval's theorem} and $v(z)$'s periodicity we get
\begin{equation*}
    N \int_0^{1/N} v^2(z) \,\mathrm{d}z = \sum_{k=-\infty}^{\infty} c_k \overline{c_k},
\end{equation*}
therefore
\begin{equation*}
    Var(v(z)) = \sum_{k=-\infty}^{\infty} c_k \overline{c_k} - \int_{0}^{1} g(h) \,\mathrm{d}h,
\end{equation*}
where $\overline{c_k}$ is the conjugate of $c_k$.\\

% Domingo  : Los limites son entre 0 y 1, no entre infinito y menos infinitos.
We introduce the following notation
\begin{equation*}
    F(u) := \int_{0}^{1} f(z) \cdot \exp{(-2\pi i u z)} \,\mathrm{d}z,
\end{equation*}
then
\begin{equation*}
    Var(v(z)) = \sum_{k=-\infty}^{\infty} F\left( k N \right) \overline{F}\left( k N \right) - \int_{0}^{1} g(h) \,\mathrm{d}h
\end{equation*}
where $\overline{F}(\cdot)$ is the conjugate of $F(\cdot)$.\\

Consider now the transform of $g(h)$, namely
\begin{multline*}
    G(u) := \int_{0}^{1} g(h) \cdot \exp{(-2\pi i u h)} \,\mathrm{d}h = \int_{0}^{1} f(x) \,\mathrm{d}x \int_{0}^{1} f(x+h) \cdot \exp{(-2\pi i u h)} \,\mathrm{d}h \\
    = \int_{0}^{1} f(x) \,\mathrm{d}x \int_{0}^{1} f(r) \cdot \exp{(-2\pi i u (r-x))} \,\mathrm{d}x = F(u) \cdot \overline{F}(u),
\end{multline*}
then the Fourier coefficients of the Variance have to be symmetric, giving,
\begin{equation*}
    Var(v(z)) = \sum_{-\infty}^{\infty} G\left( k N \right) - G(0) = 2\cdot \sum_{k=1}^{\infty} G\left( k N \right).
\end{equation*}
% Domingo: He escrito esto a toda leche.
In order to have a predictor, we are going to model the variance and suppose that the covariogram can be approximated by a polynomial, namely
\begin{equation*}
    g(h) := \sum_{j=0}^{r} a_j \cdot \abs{h}^j.
\end{equation*}
We substitute the model for the covariogram and calculate the Fourier coefficients: 
%Domingo: BUFFF, ahora me cambia lo de la gamma function. Lo cambio pero 
\begin{equation*}
    G(u) = 2\cdot \int_0^{1} \bigg( \sum_{j=0}^r a_j \cdot h^j \bigg) \cdot \exp{(-2\pi i u h)} \,\mathrm{d}h = 2\cdot \sum_{j=0}^r a_j \int_{0}^{1} h^j \cdot \exp{(-2\pi i u h)} \,\mathrm{d}h.
\end{equation*}
Because we only need real value of the variance, 
that is, the real part of $G(u)$, we define
\begin{equation*}
    \Gamma_{inc} (m) := \int_0^{1} y^{m-1} \cdot \exp{-y} \,\mathrm{d}y \approx \Gamma (m),
\end{equation*}
thus, taking $y=2\pi i u h$ we get
\begin{equation*}
    \int_{0}^{1} h^j \cdot \exp{(-2\pi i u h)} \,\mathrm{d}h = \frac{1}{(2\pi i u)^{j+1}} \int_0^{1} y^j \cdot \exp{(-y)} \,\mathrm{d}y = \frac{\Gamma (j+1)}{(2\pi i u)^{j+1}},
\end{equation*}
and because we want the real part, we only want the terms given by odd $j$, namely $j=2p-1$, $p=1,2,...$ . In consequence,
\begin{equation*}
    Var(v(z)) = 2\cdot \sum_{k=1}^{\infty} Re\left[G\left( k N \right)\right] = 2\cdot \sum_{p=1}^j a_{2p-1} \cdot \frac{1}{N^{2p}} \cdot \Bigg\{ 2\cdot \frac{\Gamma (2p)}{(-1)^p \cdot (2\pi)^{2p}} \cdot \sum_{k=1}^{\infty} \frac{1}{k^{2p}} \Bigg\}
\end{equation*}

Furthermore, since the \textit{Bernoulli number} $B_{2p} \equiv B_{2p}(0)$ can be expressed
\begin{equation*}
    B_{2p}(0) = \frac{(-1)^{p-1} \cdot 2\Gamma (2p+1)}{(2\pi)^{2p}} \cdot \sum_{k=1}^{\infty} \frac{1}{k^{2p}},
\end{equation*}
we can rewrite the variance as
\begin{equation*}
    Var(v(z)) = -\sum_{p=1}^{\lfloor \frac{r+1}{2} \rfloor} a_{2p-1} \cdot \frac{B_{2p}}{p \cdot N^{2p}} = -a_1  \frac{B_2}{N^{2}} -  a_3  \frac{B_4}{2 \cdot N^{4}} - ...
\end{equation*}

Considering that $a_1 = g'(0)$ under the polynomial model and that $B_2 = 1/6$, we can approximate the variance as follows,
\begin{equation*}
    Var(v(z)) \approx -  \frac{g'(0)}{6N^{2}}.
\end{equation*}

Finally, lets fit the parable $g(h) = a_0 + a_1 \cdot h + a_2 \cdot h^2$ through the sample points $\{ (j/N,\widehat{g}(j/N)), \hspace{1mm} j=0,1,2 \}$, where


\begin{equation*}
    \widehat{g}(j/N) = \frac{1}{N}\cdot \sum_{r=1}^{S} f_r \cdot f_{r+j}
\end{equation*}
with $S$ the total number of sections and $f_r$ the \textit{r}-th section area observed at the \textit{r}-th abscissa. More briefly, write
\begin{equation*}
    g_0 \equiv \sum f_i^2,\quad 
    g_1 \equiv \sum f_i \cdot f_{i+1},\quad
\mathrm{ and} \
    g_2 \equiv \sum f_i \cdot f_{i+2},
\end{equation*}
where $f_i \equiv f(z+i/N)$ and thus $\widehat{g}(j/N) \equiv \frac{1}{N}\cdot g_j$.
Then we have
\begin{equation*}
    \frac{1}{N} \cdot g_0 = a_0,\quad
    \frac{1}{N} \cdot g_1 = a_0 + a_1 \cdot \frac{1}{N} + a_2 \cdot \frac{1}{N^2},\quad 
    \frac{1}{N} \cdot g_2 = a_0 + a_1 \cdot \frac{2}{N} + a_2 \cdot \left(\frac{2}{N}\right)^2
\end{equation*}
from which we get
\begin{equation*}
    a_1 = \frac{4g_1 - g_2 - 3g_0}{2}.
\end{equation*}
\vspace{2mm}

Therefore, we can get the following variance estimator,
\begin{equation*}
    \widehat{Var(v(z))} \approx - \frac{a_1}{6N^2} = \frac{3g_0 + g_2 - 4g_1}{12} \cdot \frac{1}{N^2}
\end{equation*}

\vspace{2mm}

This is the general method to approximate the variance. Next, we will present the error variance estimator that has been developed for the estimation procedure that will be used in this work. The method to discover them follows the same principles but the proofs are more involved due to the number of variables.\\


%Varianza para el test system
\subsection{Variance Prediction Formula for Particle Number Estimation in the Plane with a Test System of Quadrats}

Consider $Y\subset \mathbb{R}^2$ a bounded and finite set of $N\equiv N(Y)$ particles. We want to estimate $N$ using a test system of quadrats $\Lambda_x$ whose fundamental tile $J_0$ is a square with side length $T$ and whose fundamental probe $T_0$ is a square quadrat with side length $0<t\leq T$. Let $Q$ represent the number of particles captured by $\Lambda_x$, $x\sim UR(J_0)$. Since $Y$ contains point particles, $Q$ is the total number of particles within the quadrats, otherwise, the particles should be captured using unbiased sampling rules such as the \textit{forbidden line rule}.\\

As it was shown in the first chapter, an UE for $N$ is given by
\begin{equation*}
    \widehat{N} = \frac{a}{a_0} \cdot Q = \frac{T^2}{t^2} \cdot Q
\end{equation*}
In an effort to interpret the error variance predictor for $\widehat{N}$, the test system is regarded as a two stage sample:
\begin{itemize}
    \item \textit{First stage sample:} It consists of a set of $n$ equidistant
    %%DomingoMaster: No hemos dicho que son las bandas de Cavalieri
    stripes of thickness $t$, which are called \emph{Cavalieri stripes}, and period $T$ encompassing the particle population $Y$.
    \item \textit{Second stage sample:} It consists of another set of Cavalieri stripes with the same period and thickness but perpendicular to the other one.
\end{itemize}
Using this two stage sample, we get something similar to $\Lambda_x$ providing new data such as $Q_{oi}$ and $Q_{ei}$, namely the number of particles captured by the odd and even quadrats within the \textit{i}th stripe respectively. This means that $Q_i=Q_{oi}+Q_{ei}$ is the number of particles captured by all quadrats within the \textit{i}-th stripe, therefore $Q=\sum_{i=1}^n Q_i$. As a result, a variance predictor for $\widehat{N}$ can be obtained as follows,%Importante lo de []
\begin{equation*}
    Var(\widehat{N}) = \frac{\alpha(0,\tau)}{\tau^4}\cdot [3(C_0-\widehat{\nu}_n) - 4C_1 + C_2] + \frac{\widehat{\nu}_n}{\tau^4}, \hspace{2mm} n\geq 3
\end{equation*}
where $\tau = t/T \in (0,1]$,
\begin{equation*}
    C_k = \sum_{i=1}^{n-k}Q_i Q_{i+k}, \hspace{2mm} k=0,1,2,
\end{equation*}
\begin{equation*}
    \widehat{\nu}_n = \sum_{i=1}^n Var(Q_i) = \frac{(1-\tau)^2}{3-2\tau} \cdot \sum_{i=1}^n (Q_{oi}-Q_{ei})^2
\end{equation*}
and
\begin{equation*}
    \alpha(0,\tau) = \frac{1}{6}\cdot \frac{(1-\tau)^2}{2-\tau}.
\end{equation*}
\textbf{Note:} This variance predictor must first be evaluated for a given direction of the stripes, and then for the perpendicular direction. The \textit{real variance predictor} is the average of both results (\cite{SterThAppl-2022-07-21.pdf}).




%
%
%
%
%
%
%
%
%
%
%
%
%
%
%
%
%
%
%
%
%
%
%
%
%
%
%
%
%
%
%
%
%%%%%%%%%%%%%%%%%%%%%%%%%%%%%%%%%%%%%%%%%%%%%%%%%%%%%%%%%%%%%%%%%%%%%%%%%%%%%%%%%%%%%%
%%Teoría sobre IA
\section{Artificial Intelligence Theory}
%Introducción general
%Problemas canónicos de ML %S02 Data Mining
Artificial Intelligence encompasses every method that tries to create a system that behaves as a human, this includes both of what we called \textit{true AI} and \textit{not true AI} in section \ref{121}. On the other hand, \textit{true AI} encompasses every method that gives computers the capability to learn without being explicitly programmed, which is what we call Machine Learning. This is the essential reason why Machine Learning remained as a subset of Artificial Intelligence.
Machine Learning methods can be grouped depending on their associated \textit{canonical problem}:
\begin{itemize}
    \item Clustering (Discrete Unsupervised Learning).
    \item Dimensionality Reduction (Continuous Unsupervised Learning).
    \item Classification (Discrete Supervised Learning).
    \item Prediction (Continuous Supervised Learning).
    \item Association (Association rules).
\end{itemize}
These problems can be differentiated because of their application to continuous or discrete problems, and because of the type of results/outputs to be obtained, depending on whether it requires human intervention to analyze the results (supervised) or not (unsupervised). Human intervention is needed in supervised learning because there is a target variable $Y$ which is inferred from a set of input variables $X$.
%%DomingoMaster: Otro cambio minimo.
On the contrary, unsupervised learning does not need human intervention because of its exploratory nature, as the $X$ variables are used to directly extract knowledge from the data. The lack of a clear target makes it unnecessary to output values because it is impossible to correct them as in the supervised case.
One can also talk about reinforcement learning, a type of learning that is somewhere in between supervised and unsupervised learning. Essentially, the reinforcement learning algorithm receives rewards or punishments depending on the actions performed and its consequences, trying to learn from experience (trial-and-error learning) in it's way to maximize the rewards it obtains. However, it doesn't matter what type of canonical problem we try to solve, all the methods developed to solve it can behave very poorly depending on how we decide to manage the training process.\\


%Hablar sobre overfitting y CV
Different methods may use different models inside their algorithms, but when one increases the model's complexity (i.e. number of parameters) the training process can result in overfitting. Overfitting occurs when the model is so complex that training makes it very precise, predicting almost perfectly all of the results in the training dataset. At first, having almost perfect predictions may seem like a good thing, however, those predictions result to be good only for the data in the dataset used for training. Therefore, the overfitted model gives "good predictions" for data in the training dataset, but gives poor predictions for real data. When a model predicts training data very well but real data very poorly, it is said that the model has bad generalization properties. The most important thing about a model is it's ability to generalize since we want to use the model with real data, and overfitting deteriorates the generalization capabilities of the model. To avoid overfitting, a method called cross-validation is used while training.\\

Cross-validation consists of dividing the dataset into two parts, the training dataset (containing the majority of the dataset, typically around 75\%) and the testing dataset (containing the rest of the dataset). The training dataset, as its name implies, is used to train the actual method we initially wanted to try. On the other hand, the testing dataset is used to test/evaluate the predictions of the model with data from outside the dataset that is being used to train. The train error of the model, namely the quality of the predicted results for the training dataset, will be lower the more the model is trained (this is what causes overfitting), however, the test error, namely the quality of the predicted results for the testing dataset ("real data"), will lower only until the model starts to overfit. Once the model starts to overfit, the test error begins to raise as a consequence of the model not generalizing well for "real data". Thus, the model reaches it's best generalization at the moment the test error starts to raise. It needs to be noted that the opposite of overfitting, called underfitting, should be avoided as well. For that reason, one has to set certain restrictions/thresholds so as to make sure the test error in not raising in a random iteration of the training process. One way to establish when the training of a model should stop is to establish a maximum threshold for the error difference between the train error and the test error, if the difference is higher than the threshold the training process should stop. Now that we've covered the basics of AI theory, we will unfold a more Neural Network focused theory.\\ %S02 Data Mining


\subsection{Deep Learning}
%Problemas "canonicos" de DL
As we mentioned in Chapter \ref{cap:Intro}, Neural Networks reside in a subset of Machine Learning called Deep Learning. Similarly to what happens with Machine Learning methods, Neural Networks (which are methods by themselves) can be grouped depending on their associated problems. For Neural Networks, we have the following problems:

\begin{itemize}
    \item Input-Output Supervised problems.
    \item Input-Input Supervised problems.
    \item Memory Supervised problems.
    \item Supervised and Reinforced problems.
    \item Unsupervised problems.
    \item Unsupervised/Self-Supervised Learning (Autoencoders).
\end{itemize}
%Networks de cada problema (+ desarrollo autoencoders) %S02/S08 ML1
Different Neural Network structures have been developed to solve these problems. Input-output supervised problems, where pairs $(x,y)$ for variables $X$ and $Y$ are provided, are solved using Multilayer Networks, or Feedforward Nets. These networks have several layers connected (input-hidden-output).

%IMAGENES EN IntroML1 y S08 (autoencoders)
\begin{figure}[h!]
  \begin{center}
    \includegraphics[width=90mm, height=60mm]{Figuras/MultilayerNN.png}\par
    \caption{Multilayer Neural Network scheme. Input, hidden and output layers are shown, as well as the connections between neurons from each layer.}
    \label{fig:MultilayerNN}
  \end{center}
\end{figure}

They are typically used with images, pattern recognition, interpolation and fitting. An input is given and an output is predicted using the backpropagation algorithm.\\

Input-input supervised problems, where pairs $(x,x)$ for the same variable $X$ are provided, are solved using Autoassociative memories (Hopfield). These networks have a single layer with lateral delayed connections.

\begin{figure}[h!]
  \begin{center}
    \includegraphics[width=90mm, height=60mm]{Figuras/AutoassociativeNN.png}\par
    \caption{Autoassociative memories scheme. It has only one layer, working as input, hidden and output at the same time. Lateral connections between nodes can be seen.}
    \label{fig:AutoassociativeNN}
  \end{center}
\end{figure}

They are typically used with images, pattern recognition and memory tasks. An input is given and an another input is received using the Hebbian learning.\\

Memory supervised problems, where pairs $(x,y)$ for variables $X$ and $Y$ are provided, are solved using Recurrent Networks, or Elman/Jordan nets. These networks are Multilayer Networks with hidden/output delayed lines.\vspace{5cm}

\begin{figure}[h!]
  \begin{center}
    \includegraphics[width=90mm, height=60mm]{Figuras/RecurrentNN.png}\par
    \caption{Recurrent Networks scheme. Input, hidden and output layers are shown, as well as the connections between neurons from each layer.}
    \label{fig:RecurrentNN}
  \end{center}
\end{figure}

They are typically used with video, time series analysis, natural language tasks, interpolation and fitting. An input is given and an output is predicted using the backpropagation algorithm over the time.\\

Unsupervised problems, where only the $X$ variable is provided, are solved using Competitive Networks. These networks are Multilayer Networks with lateral connections in the last layer.

\begin{figure}[h!]
  \begin{center}
    \includegraphics[width=90mm, height=60mm]{Figuras/CompetitiveNN.png}\par
    \caption{Competitive Network scheme. For this one, two layers are shown, with the last one having lateral connections between its neurons.}
    \label{fig:CompetitiveNN}
  \end{center}
\end{figure}

They are typically used for segmentation and feature extraction tasks. An input is given and clusters are provided using the ad-hoc or winner-takes-all algorithms.\\

Finally, we have Unsupervised/Self-Supervised Learning, solved using Autoencoders. Autoencoders are similar to Multilayer Networks, however, when constructing an autoencoder one doesn't really think about input-hidden-output layers, but more about encoder-decoder layers. This is because autoencoders have one specific task, learn a compressed representation of the input data (encoder) and how to reconstruct the input after having compressed it (decoder). \vspace{5cm}

\begin{figure}[h!]
  \begin{center}
    \includegraphics[width=120mm, height=50mm]{Figuras/Autoencoder.png}\par
    \caption{Autoencoder scheme. Several layers are shown. First layers conform the encoder, compressing the input data. Last layers conform the decoder, decompressing the compressed data.}
    \label{fig:Autoencoder}
  \end{center}
\end{figure}

Also, there are different autoencoder types, such as Variational Autoencoders, but they are usually normal Autoencoders with added constraints.\\

%Deep Learning que es? %S04 ML1
``Deep'' Learning does not refer to a deeper understanding of the data, but to learning successive layers of increasingly meaningful representations. The amount of network layers is referred to as the model depth, the more layers the model has the more representations can be studied and the more deep the model is. Contrary to what happens with Machine Learning, where specific representations of the data have to be obtained prior to training, with Deep Learning these representations or characteristics of the data are obtained without the need to perform previous transformations to the data. In general, Deep Learning has better performance as well, however, one has to be careful, as the Deep Learning models can overfit as well. Therefore, Cross-Validation has to be applied to Neural Networks when learning from data, similarly to what happens in Machine Learning.\\

Before going over theory related to Neural Network layers and training techniques we will cover the last few important aspects of Neural Networks: the model parameters, the activation functions, the loss function and the learning algorithm.\\ 

In Chapter \ref{cap:Intro} we introduced the notion of node (neuron) weights and biases. Weights are values used to add more or less relevancy to a certain neuron. Neurons with weight values closer to zero will have less impact on the next connected neuron and vice versa. Each neuron has different weight values for each different connection with other neurons. Also, each connection has its own bias, somewhat affecting the information transmitted to the next neuron via the linear combination of nodes and node weights. As a result, many weights and biases are employed to create a Neural Network, greatly increasing the amount of model parameters. Since weights and biases are the model's parameters, when we train we are updating their values. In general, weight and bias values are randomly initiated for the first iteration of the training process (unless the training wants to be approached in a specific manner).\\

The notion of activation function was also introduced in Chapter \ref{cap:Intro}. An activation function, $f$, is a non linear function applied to the transmitted linear combination of node and weight values, plus the bias ($f( \omega^T x + b )$, where $\omega$ is a vector of weight values, $x$ is a vector of node values and $b$ is a bias). These functions need to have easy derivatives, since computers don't know how to derive a function. Also, the derivative values of these activation functions are important, since higher derivative values imply a faster learning rate. Note that a faster learning rate is not always better, a slower or faster learning rate may be preferred depending on the type of problem since the Neural Network may or may not get stuck on local minima depending on this rate, respectively. However, low derivative values are not ideal when considering several layers since low values will be multiplied several times, giving gradient values so low that the Network won't learn (vanishing gradient problem). As a result, the choice of the activation functions is very important for Deep Learning since one may typically want a big amount of layers. One of the usual choices for the activation functions is ReLu ($f(z) = \max\{0,z\}$
%, where $z= \omega^T x + b$
) or one of its variants, as the gradient descent problem is avoided and learning is faster and less computationally expensive. However, ReLu shall not be used with Networks with a high amount of negative values since it outputs a value of zero when the input is negative, making gradients stop propagating and weights not being updated, which may cause a big part of the Network to stop learning. Different ReLu variants have been developed to target some of the ReLu problems. Another important activation function we should mention is the Softmax activation function ($g_i( z ) = \frac{\exp{z_i}}{\sum_{j=1}^{n}{\exp{z_j}}}$, where $z = (z_1, ..., z_n)$), since it is the go to activation function for the output layer in multiclass classification problems.\\

Now we will introduce the Neural Network concept of loss function. The loss function is the function the learning algorithm uses in order to check how similar the predicted results are to the real results, later updating the model's parameters accordingly. There are several different loss functions but they are all used in a similar manner. The ultimate goal is to minimize the loss function as a way to close the gap between the predicted and the real output values, meaning trying to obtain predicted values that are as similar as possible to the real output values for each sample in the training dataset. For example, for regression tasks one may use the following loss function: $L(y,\hat{y}) = - ( y \log{\hat{y}} + (1-y) \log{(1-\hat{y})} )$; where $y$ is the real output value and $\hat{y}$ is the predicted output value, whereas for multiclass classification a more general version called Cross-Entropy may be used: $L(y,\hat{y}) = - \sum_j{y_j \log{\hat{y}_j}}$; where $y$ and $\hat{y}$ are now vectors and the sum is made over the number of total categories in the dataset.\\

Finally, we will briefly talk about the learning algorithms. The most typical and basic Neural Network learning algorithm is called backpropagation. Most learning algorithms are variants of this backpropagation algorithm, so we will explain how it works to finish the explanation of the basic theory of Artificial Intelligence. As we mentioned previously, the model's parameters are randomly generated for the first iteration of the training process. Therefore, the Neuronal Network can already obtain its first predicted output values and compare them with real values using the loss function. Now is when the backpropagation algorithm is used. Backpropagation checks how much every neuron contributed to each of the predicted outputs and modifies their weights accordingly based on their respective \textit{gradient descent} values, namely based on the direction of the error gradient. Because this algorithm proceeds starting from neurons in the output layer and finishing with neurons in the input layer, the name backpropagation was associated to it. If we express this mathematically, we basically have an error function $$ E(\omega) = \frac{1}{2} \sum_{i,p}{(y_i^p - \hat{y_i^p})^2} , $$
where the sum is made over each output neuron $i$ and each sample $p$ of the training dataset, whose gradient is used to update the model's weights as $ \omega_{i,j}(t+1) = \omega_{i,j}(t) - \eta \frac{\partial E}{\partial \omega_{i,j}(t)} $, where $i,j$ is used to denote the neuron $i$ in the layer $j$, $t$ is used to denote the iteration the training process is in and $\eta$ is the learning rate we decide to set for the backpropagation algorithm. A similar process to that used with weights is then used for biases, where instead of $\omega$ we use $b$ in the equations above to obtain the updated biases. There are other versions of this backpropagation algorithm that try to improve or add a bit more complexity to the learning process, such as those which include inertia and regularization terms. However, we won't be explaining them in this work.\\

As a result, the Neural Network training process consists on a loop where we obtain predicted output values, update the model's parameters based on the prediction, and repeat.\\

%Funcionamiento de capas/tecnicas %S04/S05/S06 ML1
\subsection{Neural Network Layers/Techniques}
There are several different Neural Network layers and training techniques that one can use when developing the structure of a Neural Network. We will start showing some of the techniques used for training and finish explaining how certain Neural Network layers are used.\\

First we have regularization techniques. There are different types of regularization, but they all want to achieve the same thing, obtain lower variance values. For example, L2 regularization adds the term $\frac{\lambda}{2p}\norm{\omega}^2$ to the loss function, where $\lambda$ is the regularization parameter, making weights as small as possible and thus creating a less complex network, trying to reduce overfitting in the process.\\ 

We also have Dropout regularization, which simplifies the network by activating or deactivating a neuron with a certain set probability every time the network passes by that neuron, making it so there is not a neuron that's over-specialized, and thus trying to avoid a network that depends on a certain neuron to work properly.\\

Another type of regularization is Data Augmentation, which modifies original data in order to obtain new data, making it able to train with a bigger amount of data. More types of regularization exist, but we won't be mentioning them.\\

Some other techniques include Batch Gradient Descent, which updates the parameters after the whole dataset has been iterated, Mini-Batch Gradient Descent, which updates the parameters after each small group of dataset subdivisions is iterated, Stochastic Gradient Descent, which updates the parameters after each sample in the dataset is used, Input Normalization, which normalizes input data, Batch Normalization, which normalizes the data in each mini-batch separately, Gradient Descent With Momentum, which uses the gradient's exponential moving average to update the parameters, etc. Furthermore, some optimization algorithms with good generalization properties have been developed, such as RMSProp and Adam Optimization, which update the parameters in different ways.\\

Finally, we will talk about Neural Network layers. They are mostly used in the field of Computer Vision, which is somewhat of what Crowd Counting is about. We can differentiate between Dense Neural Networks and Convolutional Neural Networks (ConvNets). More type of Neural Networks exist, such as Residual Neural Networks, which train the network in a ``normal'' way but every certain amount of layers additionally train using residual information from past layers, however, we won't cover them. Dense Neural Networks (which are the ones we have been using so far for the theory) learn global patterns from the input feature map, they use every pixel in an image. ConvNets learn local patterns, they search in small windows (filters) of the image in 2D. What's more, ConvNets learn patterns that are translation invariant, learn the spatial hierarchy of images, are very efficient for processing images and need less images to train as they have a greater generalization power. For these reasons, ConvNets are mostly used in the field of Computer Vision.\\

First of all, we have Convolutional Layers. Images have a size in pixels, $n \times m$. Input layers need to have the same amount of neurons as the total number of pixels in the image, $n \cdot m$, however, when Neural Networks are used with images we usually represent layers as rectangles with neurons distributed as pixels in an image. Convolutional Layers are groups of ``filters'' of the image that take snips of size $a \times b$ (where $a< n$ and $b<m$ are the amount of neurons/pixels each side of the snip has, having $a \cdot b$ neurons/pixels in total per snip) and transmit all of the information in those $a \times b$ neurons/pixels to the corresponding neuron in each filter. Each filter, also named \textit{kernel}, is specialized in one task, thus, the more filters the more characteristics the Neural Network can identify.\vspace{5cm}\\

\begin{figure}[h!]
  \begin{center}
    \includegraphics[width=120mm, height=60mm]{Figuras/ConvLayer.png}\par
    \caption{Convolutional Layer scheme. The process of selecting a $5\times 5$ snip of the image and transmitting its information into a single neuron is shown for one filter.}
    \label{fig:ConvLayer}
  \end{center}
\end{figure}

Lastly, we have some techniques used with Convolutional Layers that may be described by some people as Neural Network Layers. We won't consider them as such because there are no parameters to be learned when using them. First we have Padding. Convolutional layers can make the image greatly decrease in size. Thus, constantly applying layers may make the image too small. Also, information in the pixels located at the corner of the image are underrepresented in the feature maps. Padding consists of adding a ``frame'' with a width of $r$ pixels surrounding the image in order to avoid these problems. When Padding is not used one may call it ``Valid'' Convolution. If the Padding is applied until the input and output have the same size one may call it ``Same'' Convolution. Next we have Strided Convolution. This consists of moving the snips in steps of $s$ pixels instead of moving it in steps of 1 pixel. Lastly, we have Pooling. Pooling is used to reduce the dimensionality of the feature map of the image. It splits the feature map into groups with the same size $d \times d$ and then applies a specific function to each group, creating a new layer with only one value per group. Some other techniques are used to go from the convolutional part to the classification part of a network, like Flatten, which puts all layers into a single flattened array (vector), or Global Average Pooling, which uses Pooling to reduce the layers into a single pixel. There may be more techniques out there, but we have covered the more basic ones.\\

%Sliding window
There is, however, one slightly more complex technique that we will mention because it is used during the execution of our Crowd Counting model, the Sliding Window Technique. Simply put, it consists of creating a small window of fixed size in the top-left corner of the image and sliding that window across the image systematically. For each window position the model is used within the window part of the image, obtaining the results and lastly combining them. It should be mentioned that the sliding is done in such a way that the whole image could be recreated by setting the windows side to side. However, in some cases there may be an overlap between nearby windows to ensure that no details are missed. There are several versions of the Sliding Window Technique, but we only want to give you an idea of how it works, so no further details will be provided\footnote{(\url{https://supervisely.com/blog/how-sliding-window-improves-neural-network-models/})}.\\ %(\cite{sliding}).
%CREE UNA CITA PARA ESTA URL PERO SUPONGO QUE EL FOOTNOTE ES MEJOR (COMO PUSISTE PARA kdnuggets EN LA INTRO


It is worth mentioning that the most general structure of Convolutional Neural Networks consists of alternatively using Convolutional Layers and Pooling until we have many feature maps with low spatial size. Then, the last feature maps are flattened and Dense Layers are applied until it obtains the output. After that, softmax would be applied it we were doing classification.\\



%
%
%
%
%
%
%
%
%
%
%
%
%
%
%
%
%
%
%
%
%
%
%
%
%
%
%
%
%
%
%
%
%Problemas de la IA en detección de personas
%%DomingoMaster: Suena fuerte.
\subsection{Challenges in Crowd Counting}
There are several challenges concerned with automatic estimation of the number of individuals in images or videos. We will go over some of them and establish what CLIP-EBC does to try and solve those issues.\\

We will start off with classification-based methods. Classification-based crowd-counting methods have inappropriate discretization strategies which impede the application of CLIP and lead to suboptimal performance. Another limitation of classification-based methods is their sole focus on the classification error without considering the proximity of predicted count values to the ground truth (the ``real'' results). This compromises performance in testing, as two probability distributions with identical classification errors may exhibit different expectations (\cite{CLIP}).\\

State-of-the-art crowd counting methods use training images with dot annotations of individuals' head centers, but they train with a density map estimation, losing the individuals' precise location. Also, since the training images are sparse binary matrices and density maps are dense real-valued matrices, a function defined based on the pixel-wise difference between the annotated and predicted density maps is hard to train because the reconstruction loss is heavily unbalanced between the 0s and 1s in the sparse binary matrix. To alleviate this problem each annotated dot is turned into a Gaussian Blob using Gaussian kernels, however, the performance of the resulting network is highly dependent on the quality of this ``pseudo ground truth'', plus, using Gaussian kernels exacerbates the noise during the annotation process. Moreover, this arises two major problems: 
\begin{itemize}
    \item First, given that individuals are often depicted at different scales in images due to perspective distortion, an ideal scenario would involve matching kernel widths with head sizes, which are not provided for counting tasks. Thus, we need to set the kernel widths for each annotated dot to construct the likelihood function using the Gaussian kernels. If the kernel size is set too small, pixels corresponding to individuals' heads are set to 0 in the density map. Conversely, if the kernel size is too large, pixels corresponding to the background can be mistaken as pedestrians. % To address this issue, Wang et al.[6] introduce the DMCount loss by leveraging the discrete optimal transport theory. This loss function does not require Gaussian smoothing and models trained with it can have enhanced performance.
    \item Second, the loss corresponds to an underdetermined system of equations with infinitely many solutions.
\end{itemize}
Gaussian smoothing also transforms the initially discrete count values into a continuous space $[0,\infty)$, needing to use a sequence of bordering real-valued intervals $(a,b]$ as bins, which makes samples near the borders challenging to classify, thus making it difficult for models to learn optimal decision boundaries.
Furthermore, methods that use this approach assume the crowd is evenly (uniformly) distributed, when in reality crowd distribution is quite irregular. All of this hurts generalization performance, specially struggling with high-density images (\cite{CLIP},\cite{DMCount},\cite{FGENet}).\\

A majority of these methods adopt an encoder-decoder framework, aiming to directly regress the density maps. They typically output density maps with reduced spatial sizes determined by a reduction factor. Each element in the density map estimates the count value in a corresponding block of the image. However, these methods overlook that count values exhibit a long tail distribution, where areas with large values are severely undersampled. To address this challenge some methods (such as CLIP-EBC) turn crowd counting into a classification task by merging count values into bins (classes), therefore, the sample sizes of rare values can be increased. These methods are also based on blockwise prediction but output probability maps of reduce spatial sizes, where the vector at each spatial location represents the probability scores over the bins. The predicted density map is calculated by aggregating mean values of bins, each weighted according to the associated probability score. The final predicted count is then derived by integrating the resulting density map (\cite{CLIP}).\\

Point frameworks can solve issues caused by the density map framework, but they cannot avoid the noise issue introduced in the annotation process. Both label noise and missing mark can ruin the quality of ground truth introducing inaccuracies and inconsistencies in the ground truth data, thus impeding the accurate estimation of crowd counting. Moreover, the preservation of fine-grained information remains a critical concern (\cite{FGENet}).\\

As a result, CLIP's capability to count has to deal with two primary challenges:
\begin{itemize}
    \item The inherent mismatch between CLIP, designed for recognition, and counting, which is a regression task.
    \item The limitations and suboptimal results of existing classification-based counting methods.
\end{itemize}
The EBC framework was specifically designed to address the challenges faced by classification-based methods, relying on integer-valued bins that facilitate the learning of robust decision boundaries. However, with few modifications, regression-based methods can be integrated into the EBC framework, enhancing their performance (\cite{CLIP}).\\

Finally, it is worth mentioning some of the problems that every Neural Network has to face when dealing with crowd counting problems:
\begin{itemize}
    \item Diversity: The huge diversity in individuals' appearances and their assemblies. This includes poses, view-points and illumination variations within the crowd and across crowd images.
    \item Scale: The extreme scale and density variations in crowds. This includes the variations in head sizes due to perspective distortion.
    \item Resolution: Limited context to discriminate people in feature maps with higher resolution. For example, patterns formed by leaves, structures of buildings, cluttered backgrounds, etc. may resemble a formation of people in high-density crowd scenarios.
    \item Data imbalance: The presence of a majority of images with certain characteristics and few images with different characteristics to that of the latter ones can cause problems while training.
    \item Local minima: Training with higher resolutions increases the chances of optimization being stuck in local minima, leading to suboptimal performance, especially with diverse crowd data.
\end{itemize}
(\cite{LSC-CNN}).



%COSAS DE LAS QUE HABLAR AL MENCIONAR PROBLEMAS
%VGG-16 convolution layers for better crowd feature extraction. (LSC-CNN)
%no creo que lo incluya, creo que no se usa





%Funcionamiento del modelo
\subsection{The CLIP-EBC Model}
%QUIZAS DEBERIA INCLUIR VARIAS IMAGENES DE CLIP, SON MUY BUENAS
The Contrastive Language-Image Pre-training (CLIP) model has demonstrated outstanding performance in tasks such as zero-shot image classification and object detection, however, there are only a few studies in crowd counting that use CLIP. Nonetheless, crowd counting methods that use CLIP are either not able to generate density maps or rooted in density-map regression, leading to suboptimal performance in crowd counting tasks. CLIP-EBC is the first fully CLIP-based crowd-counting model capable of generating density maps. In this subsection we will summarize the procedure used in \cite{CLIP}. For more details about the CLIP-EBC model head over to their paper.\\

\subsubsection{Enhanced Blockwise Classification}
The EBC framework resorts to blockwise prediction using a probability map $\mathbf{P}*$ to predict the density map $\mathbf{Y}*$ at the pixel level, in spite of the inherent noise present in point labels. However, contrary to regression-based methods (which suffer from undersampling of large count values), EBC groups count values into bins to increase the sample size of each bin, alleviating the sample imbalance problem. As a result, let $\{ \mathcal{B}_i \hspace{2mm} |\hspace{2mm} i=1, ... ,n;\hspace{2mm} \mathcal{B}_i \cap \mathcal{B}_j \hspace{2mm} \forall i \neq j;\hspace{2mm} \mathcal{S} \subset \cup_{i=1}^{n} \mathcal{B}_i \}$ be the $n$ pre-defined bins, where $\mathcal{S}$ is the support set of count values. For an image $\mathbf{X} \in \mathbb{R}_{+}^{C\times H\times W}$, where $C$ represents the number of channels and $H$ and $W$ represent the spatial height and width of the image respectively, the predicted density map can be obtained as follows:
$$ \mathbf{Y}_{i,j}^* = \sum_{k=1}^{n} a_k \cdot \mathbf{P}_{k,i,j}^* , $$ where $a_k$ is a representative count value, the probability map $\mathbf{P}^*$ has dimensions $(n, H//r, W//r)$, where $//$ represents the floor division operator, and $\mathbf{P}_{:,i,j}^*$ denotes the probability scores of the bins in the region $(r(i-1):ri, r(j-1):rj)$ of the image. Summing over $\mathbf{Y}_{i,j}^*$ would result in the predicted count for the whole image.\\

With respect to the Gaussian smoothing issues, EBC bypasses Gaussian smoothing and adopts a YOLO-like approach. If an individual is within a specific block, EBC compels only that block to predict the presence of that individual while excluding other blocks from making such predictions. This strategy helps preserve the inherent discreteness of the count. Furthermore, three bin strategies of varying granularity can be used: \textit{fine} (each bin contains one integer, providing bins with the lowest biases), \textit{dynamic} (creates bins of various sizes, considering small count values as individual bins and combining every two for larger count values), and \textit{coarse} (each bin comprises more than one integer, acknowledging the long-tail distribution of count values and increasing sample sizes of each bin). In \cite{CLIP} it was shown that, in general, dynamic granularity provides the best performance (as it achieves a good balance between reducing biases in count values and increasing sample sizes), followed by fine granularity and lastly coarse granularity. Also, to account for the representative count values not following a uniform distribution, EBC uses the average count values in each bin as the representative point:
$$ a_i = \frac{1}{\abs{\mathcal{B}_i}} \sum_{k=1}^{M}{\mathbbm{1}(c_k \in \mathcal{B}_i) \cdot c_k} , $$
where $\abs{\mathcal{B}_i}$ is the cardinality of the bin $\mathcal{B}_i$, $M$ is the number of all blocks in the dataset, $\mathbbm{1}$ is the indicator function and $c_k$ is the count value in block $k$.\\

What's more, annotations within densely populated image areas can be exceedingly erroneous and noisy, giving models incorrect backpropagation signals and degrading their performance. EBC proposes to constrain the maximum count of observable people in fixed-size image patches to a small constant determined by the patch size. Specifically, it posits that the minimum recognizable size for a person is $l \times l$ pixels, thus, the maximum allowable count value is $\mathcal{M} = (r//l)^2$, where $r$ is a model-related reduction factor.\\

%Funciones de pérdida
When it comes to the Neural Network loss function, EBC uses the Distance-Aware-Cross-Entropy (DACE) loss proposed in \cite{CLIP}:
\begin{eqnarray}\label{loss}
\mathcal{L}_{DACE} & = & \mathcal{L}_{class}(\mathbf{P}^*,\mathbf{P}) + \lambda \mathcal{L}_{count}(\mathbf{Y}^*,\mathbf{Y}) \\
             & = & - \sum_{i=1}^{H//r}\sum_{j=1}^{W//r}\sum_{k=1}^{n} \mathbbm{1}(\mathbf{P}_{k,i,j} = 1) \log{\mathbf{P}_{k,i,j}^*} + \lambda \mathcal{L}_{count}(\mathbf{Y}^*,\mathbf{Y}),
\end{eqnarray} 
where $\mathbf{P}$ is the one-hot encoded ground-truth probability map, $\mathbf{P}^*$ is the predicted probability map, $\mathbf{Y}$ is the ground-truth density map, $\mathbf{Y}^*$ is the predicted density map and $\mathcal{L}_{count}$ is the loss count function weighted by $\lambda$. The loss count function can be any function that measures the difference between two density maps. This framework (EBC) uses the DMCount Loss function [\cite{DMCount}] since the ground truth is not Gaussian smoothed (thus, avoiding Gaussian kernel's problems).\\

\subsubsection{The Structure of CLIP-EBC}
The image encoder of CLIP-EBC includes a feature extractor and a $1\times 1$ Convolutional Layer. As the model is based on blockwise prediction, the final pooling layer and linear projection layer are removed. The remaining backbone is used to extract the feature map instead. After that, it employs a $1\times 1$ Convolutional Layer to transform the feature map into the CLIP embedding space, yielding the image feature maps.\\

For the text feature extraction, CLIP-EBC generates one text prompt for each bin according to the following rules:
\begin{itemize}
    \item If the bin has only one count value $q$ such that $q<\mathcal{M}$, the text prompt is: ''There is/are $q$ person/people´´.
    \item If the bin has more than one element, let $u,v$ denote the minimum and maximum values of the bin, respectively. Then the text prompt is: ''There is/are between $u$ and $v$ person/people´´.
    \item If the bin has only one count value $q$ such that $q=\mathcal{M}$, the text prompt is: ''There are more than $\mathcal{M}$ people´´.
\end{itemize}
Subsequently, the resulting $n$ text prompts are tokenized by CLIP's tokenizer, the tokenized text is set as input in the CLIP text encoder and the text features are yielded.\\

With the image feature maps and text features CLIP-EBC can then obtain the probability map $\mathbf{P}^*$. First, it calculates the cosine similarity between the image feature vector in a set position of the image and the $n$ extracted text features.  %dimensiones mal? %???????????????????????????????????????????????????????????????
Next, similarities are normalized using softmax to obtain the probabilities $\mathbf{P}_{:,i,j}^*$. Finally, the predicted density map is obtained using the equations mentioned above.\\





%
%
%
%
%
%
%
%
%
%
%
%
%
%
%
%
%
%
%
%
%
%
%
%
%
%
%
%
%
%
%
%
%%%%%%%%%%%%%%%%%%%%%%%%%%%%%%%%%%%%%%%%%%%%%%%%%%%%%%%%%%%%%%%%%%%%%%%%%%%%%%%%%%%%%%
%%Integrar IA + Estereología
\section{Combining AI and Stereology}
%intro breve
In this work we aim at combining a Crowd-Counting Neural Network model with the mathematical discipline of Stereology. As we mentioned, we are going to use the CLIP-EBC model for this purpose, however, the basis on which our estimation method is based can be adapted so as to be employed in a similar manner for many different models and problems.\\

In order to explain how our estimation method works we will first introduce the notion of Quasi-Monte Carlo Integration and Optimal Point Sets. We will later see how these Point Sets can be used for the purpose of this project.\\



%
%
%
%
%
%
%
%
%
%
%
%
%
%
%
%
%
%
%
%
%
%
%
%
%
%
%
%
%
%
%
%
%
%
%
%
\subsection{Quasi-Monte Carlo Integration (QMC Integration)}
%Forma de calcular las esperanzas a partir de sumas en vez de integrales. También objetivo de lo que se pretende (mejorar la eficiencia).
%%%%%%%%%%%%%%%%%%%%%%%%%%%%%%%%%%%%%%%%%%%%%%%%%%%%%%%%%%%%%%%
% Samplings points in quasi-montecarlo integration method. (Hinrich.pdf) 
% Spaces of functions (https://en.wikipedia.org/wiki/Reproducing_kernel_Hilbert_space). 
% Error depending 

One problem that has to be dealt with in order to compute a target measure estimate is the numerical integration of multivariate functions. This problem arises because of the need to obtain the expectation for the number of individuals in our estimation problem.\\

To simplify things, lets normalize the integration domain to be the compact unit cube $[0,1]^d$, that is,
\begin{equation*}
    \int_{[0,1]^d} f(\textbf{x})\,\mathrm{d}\textbf{x} = \int_0^1 \cdot \cdot \cdot \int_0^1 f(x_1,...,x_d)\,\mathrm{d}x_1\ldots\mathrm{d}x_d
\end{equation*}
The goal is to approximate these integrals using \textit{QMC rules} with fixed integration nodes $\textbf{x}_0,...,\textbf{x}_{N-1} \in [0,1)^d$, namely
\begin{equation} \label{QMC}
    \int_{[0,1]^d} f(\textbf{x})\,\mathrm{d}\textbf{x} \approx \mathcal{Q}_{N,d}(f) := \frac{1}{N} \sum_{n=0}^{N-1} f(\textbf{x}_n).
\end{equation}
\vspace{2mm}

The crux of this method is the choice of underlying nodes. What's more, we also need some global information on the functions to be integrated, since one can find two functions $f,g : [0,1]^d \longrightarrow \mathbb{R}$ such that $f(\textbf{x}_n) = g(\textbf{x}_n) \hspace{2mm} \forall n=0,1,...,N-1$, but $\int_{[0,1]^d} f(\textbf{x})\,\mathrm{d}\textbf{x} - \int_{[0,1]^d} g(\textbf{x})\,\mathrm{d}\textbf{x}$ can be any number. So as to avoid this problem, function classes with certain smoothness properties are considered.\\

Lets start with \textit{univariate QMC integration}, namely QMC integration of univariate real valued functions $f:[0,1] \longrightarrow \mathbb{R}$ with continuous first derivative on $[0,1]$. From the \textit{fundamental theorem of calculus} we have
\begin{equation*}
    f(x) = f(1) - \int_x^1 f'(y)\,\mathrm{d}y, \hspace{2mm} \forall x \in [0,1].
\end{equation*}
For the error of a QMC rule based on sample nodes $\mathcal{P}=\{ x_0,...,x_{N-1} \} \subset [0,1)$ we then get
\begin{multline*}
    e(f,\mathcal{P}) = \int_0^1 f(x)\,\mathrm{d}x - \frac{1}{N} \sum_{n=0}^{N-1} f(x_n) = -\int_0^1 \int_x^1 f'(y)\,\mathrm{d}y \,\mathrm{d}x + \frac{1}{N} \sum_{n=0}^{N-1} \int_{x_n}^1 f'(y) \,\mathrm{d}y \\
    = -\int_0^1 \int_0^y f'(y)\,\mathrm{d}x \,\mathrm{d}y + \int_0^1 \frac{1}{N} \sum_{n=0}^{N-1} 1_{(x_n,1]}(y) f'(y) \,\mathrm{d}y = \int_0^1 f'(y) \Bigg[\frac{1}{N} \sum_{n=0}^{N-1} 1_{(x_n,1]}(y) - y\Bigg] \,\mathrm{d}y
\end{multline*}
Since the number of indices $n\in \{ 0,...,N-1 \}$ for which $x_n \in [0,y)$ is
\begin{equation*}
    \sum_{n=0}^{N-1} 1_{(x_n,1]}(y) = \sum_{n=0}^{N-1} 1_{(0,y]}(x_n)
\end{equation*}
we find that
\begin{equation*}
    \frac{1}{N} \sum_{n=0}^{N-1} 1_{(x_n,1]}(y) - y = \Delta_{\mathcal{P},N}(y),
\end{equation*}
that is, the local discrepancy of $\mathcal{P}$ in $y$. Therefore,
\begin{equation*}
    e(f,\mathcal{P}) = \int_0^1 f'(y)\Delta_{\mathcal{P},N}(y)\,\mathrm{d}y.
\end{equation*}

Taking the absolute value and applying the \textit{triangle inequality for integrals} and the \textit{Hölder inequality} we get
\begin{equation*}
    \abs{e(f,\mathcal{P})} \leq \int_0^1 \abs{f'(y)} \abs{\Delta_{\mathcal{P},N}(y)} \,\mathrm{d}y \leq \left( \int_0^1 \abs{f'(y)}^r \,\mathrm{d}y \right)^{1/r} \left(  \int_0^1 \abs{\Delta_{\mathcal{P},N}(y)}^s \,\mathrm{d}y \right)^{1/s} \\
    = \norm{f'}_{L_r} \norm{\Delta_{\mathcal{P},N}}_{L_s},
\end{equation*}
where $r,s \in [0,\infty]$ such that $\frac{1}{r} + \frac{1}{s} = 1$. For QMC integration of functions $f$ for which $\norm{f'}_{L_r} < \infty$, sample nodes $\mathcal{P}$ with low $L_s$ discrepancy $L_{s,N}(\mathcal{P}) = \norm{\Delta_{\mathcal{P},N}}_{L_s}$ should be chosen.\\

\textbf{Note:}
In order to develop a similar theory for multivariate functions, the notion of \textit{reproducing kernel Hilbert space} will be used.
For an integrable function $f : [0,1]^d \longrightarrow \mathbb{C}$ and a point set $\mathcal{P} = \{\textbf{x}_0,...,\textbf{x}_N-1\} \subset [0,1)^d$, the notation:
\begin{equation*}
    e(f,\mathcal{P}) = \int_{[0,1]^d} f(\textbf{x}) \,\mathrm{d}\textbf{x} - \frac{1}{N} \sum_{n=1}^{N-1} f(\textbf{x}_n),
\end{equation*}
will be used in analogy to the univariate case. What's more, the inner product in a \textit{Hilbert Space} $\mathcal{H}$ will be denoted by $\langle \cdot,\cdot \rangle$ and the corresponding norm will be denoted by $\norm{\cdot} = \langle \cdot,\cdot \rangle^{1/2}$.\\


Now lets give some useful definitions used with Hilbert spaces in order to develop a similar theory for multivariate functions.\\

\begin{Def}
    The \textit{worst-case error} of a QMC rule based on a point set $\mathcal{P} = \{\textbf{x}_0,...,\textbf{x}_N-1\} \subset [0,1)^d$ in a Hilbert space $\mathcal{H}$ of integrable functions on $[0,1]^d$ is defined as
    \begin{equation*}
        e(\mathcal{H},\mathcal{P}) = \underset{f \in \mathcal{H}, \norm{f}\leq 1}{\text{sup}} \abs{e(f,\mathcal{P})}.
    \end{equation*}
\end{Def}

\vspace{2mm}
Also, the notion of \textit{reproducing kernel} will be important for our purposes.\\

\begin{Def}
    A Hilbert space $\mathcal{H}$ of functions on $[0,1]^d$ is a \textit{reproducing kernel Hilbert space} on $[0,1]^d$ if there exists a function $K : [0,1]^d \times [0,1]^d \longrightarrow \mathbb{C}$ such that
    \begin{itemize}
        \item[K1:]\label{K1} $K(\cdot,\textbf{y}) \in \mathcal{H} \hspace{2mm} \forall \textbf{y} \in [0,1]^d$ 
        \item[K2:]\label{K2} $\langle f,K(\cdot,\textbf{y}) \rangle = f(\textbf{y}) \hspace{2mm} \forall \textbf{y} \in [0,1]^d, \hspace{1mm} \forall f\in \mathcal{H}$.
    \end{itemize}
    The function $K$ is the \textit{reproducing kernel} of $\mathcal{H}$.\\
\end{Def}

\textbf{Note:} the reproducing kernel $K$ has been considered as a function of the first variable, denoted by $\cdot$, thus, in $\langle f,K(\cdot,\textbf{y}) \rangle$ the inner product is applied with respect to the first variable of $K$.\\


Furthermore, reproducing kernels have the following properties.\\

\begin{Corollary}
    A function satisfying K1 and K2 is symmetric, uniquely defined and positive semi-definite, that is,
    \begin{itemize} 
        \item[K3:]\label{K3} $K(\textbf{x},\textbf{y}) = \overline{K(\textbf{y},\textbf{x})} \hspace{2mm} \forall \textbf{x},\textbf{y} \in [0,1]^d$ (symmetry) 
        \item[K4:]\label{K4} If a function $\Tilde{K}$ satisfies K1 and K2, then $\Tilde{K} = K$ (uniqueness)
        \item[K5:]\label{K5} For any $a_0,...,a_{N-1} \in \mathbb{C}$ and $\textbf{x}_0,...,\textbf{x}_{N-1} \in [0,1]^d$, we have $\sum_{m,n=0}^{N-1} \Tilde{a}_m a_n K(\textbf{x}_m,\textbf{x}_n) \geq 0$ (positive semi-definiteness) 
    \end{itemize}
\end{Corollary}
\begin{proof} K3: $K(\textbf{x},\textbf{y}) = \langle K(\cdot,\textbf{y}),K(\cdot,\textbf{x}) \rangle = \overline{\langle K(\cdot,\textbf{x}),K(\cdot,\textbf{y}) \rangle} = \overline{K(\textbf{y},\textbf{x})} $.\\
K4: $\Tilde{K}(\textbf{x},\textbf{y}) = \langle \Tilde{K}(\cdot,\textbf{y}),K(\cdot,\textbf{x}) \rangle = \overline{\langle K(\cdot,\textbf{x}),\Tilde{K}(\cdot,\textbf{y}) \rangle} = \overline{K(\textbf{y},\textbf{x}) = K(\textbf{x},\textbf{y})}$.\\
K5: 
\begin{multline*}
    \sum_{m,n=0}^{N-1} \Tilde{a}_m a_n K(\textbf{x}_m,\textbf{x}_n) = \sum_{m,n=0}^{N-1} \Tilde{a}_m a_n \langle K(\cdot,\textbf{x}_n),K(\cdot,\textbf{x}_m) \rangle = \langle \sum_{n=0}^{N-1} a_n K(\cdot,\textbf{x}_n),\sum_{m=0}^{N-1} a_m K(\cdot,\textbf{x}_m) \rangle \\
    = \begin{Vmatrix} \sum_{m=0}^{N-1} a_m K(\cdot,\textbf{x}_m) \end{Vmatrix}^2 \geq 0.
\end{multline*} 

This concludes the proof.
\end{proof}

\vspace{2mm}
Additionally, a function $K$ satisfying K3 and K5 uniquely determines a Hilbert space of functions  for which K1, K2 and K4 hold. Whereby, it makes sense to speak of a reproducing kernel without specifying the corresponding Hilbert space.\\

That being said, let $\mathcal{H}$ be a Hilbert space of integrable functions $f : [0,1]^d \longrightarrow \mathbb{C}$ equipped with inner product $\langle \cdot,\cdot \rangle$ and corresponding norm $\norm{\cdot} = \langle \cdot,\cdot \rangle^{1/2}$.\\

\begin{Def}
    A linear functional $T$ on $\mathcal{H}$ is \textit{bounded} if there exists $M<\infty$ such that $\abs{T(f)}\leq M$ for all $f$ with $\norm{f}\leq 1$.
\end{Def}

\vspace{2mm}
Also, it is known that linear functionals boundedness is equivalent to continuity.\\

Lets consider $e(\mathcal{H},\mathcal{P})$ (\textit{i.e.} the worst-case error of a QMC rule based on a point set $\mathcal{P}\subset [0,1)^d$ in $\mathcal{H}$). It's not clear if $e(\mathcal{H},\mathcal{P})$ is finite, that is to say, if the linear functional $e(\cdot,\mathcal{P})$ is continuous. However, conditions exist such that this is the case.\\

Let $T_y$ be the linear functional that evaluates $f\in \mathcal{H}$ at $\textbf{y}\in [0,1]^d$, namely
\begin{equation*}
    T_y(f) = f(\textbf{y}) \hspace{2mm} \text{for } f\in \mathcal{H}.
\end{equation*}
$T_y$ is called the \textit{evaluation functional} in $y$. It turns out that if $T_y$ is continuous $\forall \textbf{y} \in [0,1]^d$, then so is every QMC rule. What's more, continuity of the evaluation functional is equivalent to the existence of a reproducing kernel.\\

\begin{Th}
    Let $\mathcal{H}$ be a Hilbert space of functions on $[0,1]^d$. $\mathcal{H}$ is a reproducing kernel Hilbert space on $[0,1]^d$ if and only if the evaluation functionals $$ T_y(f) = f(\textbf{y}) \hspace{2mm} \text{for } f \in \mathcal{H}, \textbf{y} \in [0,1]^d $$ are continuous.
\end{Th}
\begin{proof} $\Leftarrow)$ We assume that the evaluation functionals are continuous, therefore, the \textit{Fréchet-Riesz Representation Theorem} guarantees that, for every $\textbf{y}$, a uniquely determined function $k_y \in \mathcal{H}$ exists such that $$ T_y(f) = \langle f,k_y \rangle \hspace{2mm} \forall f \in \mathcal{H}.$$ Defining $K(\textbf{x},\textbf{y}) := k_y(\textbf{x})$ for $\textbf{x},\textbf{y} \in [0,1]^d$, properties K1 and K2 are satisfied by $K$, thus $\mathcal{H}$ is a reproducing kernel Hilbert space whose reproducing kernel is $K$.\\ 

$\Rightarrow)$ Assume $K$ is a reproducing kernel for $\mathcal{H}$ and let $\textbf{y} \in [0,1]^d$. Using the Cauchy-Schwarz inequality, we get $$ \abs{T_y(f)} = \abs{f(\textbf{y})} = \abs{\langle f,K(\cdot,\textbf{y}) \rangle} \leq \norm{f} \norm{K(\cdot,\textbf{y})} \hspace{2mm} \forall f \in \mathcal{H}.$$
Now, using K2 we have $\norm{K(\cdot,\textbf{y})}^2 = \langle K(\cdot,\textbf{y}),K(\cdot,\textbf{y}) \rangle = K(\textbf{y},\textbf{y})$, whereby $\abs{T_y(f)}\leq M := \sqrt{K(\textbf{y},\textbf{y})}$ for every $f$ with $\norm{f}\leq 1$. This means that $T_y$ is continuous, which concludes the proof. %Pregunta????
\end{proof}

\vspace{2mm}
Continuing with the situation above, lets consider the integration functional $I(f) = \int_{[0,1]^d} f(\textbf{x})\,\mathrm{d}\textbf{x}$. If $\mathcal{H}$ has a reproducing kernel $K$, then for any $f \in \mathcal{H}$ with $\norm{f}\leq 1$ we have
\begin{equation*}
    \abs{\int_{[0,1]^d} f(\textbf{y}) \,\mathrm{d}\textbf{y}} = \abs{\int_{[0,1]^d} T_y(f) \,\mathrm{d}\textbf{y}} \leq \int_{[0,1]^d} \abs{T_y(f)} \,\mathrm{d}\textbf{y} \leq \int_{[0,1]^d} \sqrt{K(\textbf{y},\textbf{y})} \,\mathrm{d}\textbf{y}.
\end{equation*}
Therefore, if the reproducing kernel $K$ satisfies
\begin{itemize}
    \item[$\textbf{C}$:] $\int_{[0,1]^d} \sqrt{K(\textbf{y},\textbf{y})} \,\mathrm{d}\textbf{y} < \infty$,
\end{itemize}
then the integration functional $I$ is continuous.\\

In conclusion, if $\mathcal{H}$ has a reproducing kernel $K$ that satisfies $\textbf{C}$, then function evaluation and integration are continuous linear functionals, and so is $e(\cdot,\mathcal{P})$ for any point set $\mathcal{P}$. What's more, under these conditions $e(\mathcal{H},\mathcal{P})$ is a well-defined finite number.\\

A very important property for the error analysis follows.\\

\begin{Lemma}
    Let $\mathcal{H}$ be a reproducing kernel Hilbert space with reproducing kernel $K$ and inner product $\langle \cdot,\cdot \rangle$. If the mapping
    \begin{equation*}
        I(f) = \int_{[0,1]^d} f(\textbf{y}) \,\mathrm{d}\textbf{y} \hspace{2mm} \text{for } f\in \mathcal{H}
    \end{equation*}
    is a continuous linear functional on $\mathcal{H}$, then
    \begin{equation*}
        \int_{[0,1]^d} \langle f,K(\cdot,\textbf{y}) \rangle \,\mathrm{d}\textbf{y} = \langle f,\int_{[0,1]^d} K(\cdot,\textbf{y}) \,\mathrm{d}\textbf{y} \rangle.
    \end{equation*}
\end{Lemma}
\begin{proof} We assume that $I$ is continuous, therefore, the Fréchet-Riesz representation theorem guarantees the existence of a uniquely determined function $R \in \mathcal{H}$ such that
\begin{equation*}
    \int_{[0,1]^d} f(\textbf{y}) \,\mathrm{d}\textbf{y} = I(f) = \langle f,R \rangle \hspace{2mm} \forall f\in \mathcal{H}.
\end{equation*}
Given that $R\in \mathcal{H}$, we have
\begin{equation*}
    R(\textbf{x}) = \langle R,K(\cdot,\textbf{x}) \rangle = \overline{\langle K(\cdot,\textbf{x}),R \rangle} = \overline{\int_{[0,1]^d} K(\textbf{y},\textbf{x}) \,\mathrm{d}\textbf{y}}.
\end{equation*}
Whereby
\begin{multline*}
    \int_{[0,1]^d} \langle f,K(\cdot,\textbf{y}) \rangle \,\mathrm{d}\textbf{y} = \int_{[0,1]^d} f(\textbf{y}) \,\mathrm{d}\textbf{y} = \langle f,R \rangle = \langle f,\overline{\int_{[0,1]^d} K(\textbf{y},\cdot) \,\mathrm{d}\textbf{y}} \rangle \\
    = \langle f,R \rangle = \langle f,\int_{[0,1]^d} K(\cdot,\textbf{y}) \,\mathrm{d}\textbf{y} \rangle.
\end{multline*}

This concludes the proof.
\end{proof}

\textbf{Note:} From now on, $K$ will be assumed to satisfy the condition \textbf{$C$}.\\

Lets continue the worst-case error analysis. We have
\begin{equation*}
    I(f) = \int_{[0,1]^d} \langle f,K(\cdot,\textbf{y}) \rangle \,\mathrm{d}\textbf{y} = \langle f,\int_{[0,1]^d} K(\cdot,\textbf{y}) \,\mathrm{d}\textbf{y} \rangle.
\end{equation*}
On the other hand, we have
\begin{equation*}
    \mathcal{Q}_{N,d}(f) = \frac{1}{N} \sum_{n=0}^{N-1} f(\textbf{x}_n) = \frac{1}{N} \sum_{n=0}^{N-1} \langle f,K(\cdot,\textbf{x}_n) \rangle = \langle f,\frac{1}{N} \sum_{n=0}^{N-1} K(\cdot,\textbf{x}_n) \rangle.
\end{equation*}
Therefore, the QMC rule $\mathcal{Q}_{N,d}$ in $\mathcal{H}$ has an integration error that can be expressed as an inner product, namely
\begin{equation*}
    e(f,\mathcal{P}) = \langle f,h \rangle,
\end{equation*}
where
\begin{equation*}
    h(\textbf{x}) = \int_{[0,1]^d} K(\textbf{x},\textbf{y}) \,\mathrm{d}\textbf{y} - \frac{1}{N} \sum_{n=0}^{N-1} K(\textbf{x},\textbf{x}_n).
\end{equation*}
The function above is called the \textit{representer} of the integration error. Taking the absolute value and applying the Cauchy-Schwarz inequality, we get
\begin{equation*}
    \abs{e(f,\mathcal{P})} \leq \norm{f} \cdot \norm{h}.
\end{equation*}
What's more, among all functions in the unit ball of $\mathcal{H}$, the normalized representer (\textit{i.e.} $h/\norm{h}$) is the hardest to integrate. Thus, the worst-case error reads
\begin{equation*}
    e(\mathcal{H},\mathcal{P}) = \norm{h}.
\end{equation*}
As a result, the squared worst-case error can be written as
\begin{equation*}
    e^2(\mathcal{H},\mathcal{P}) = \langle h,h \rangle.
\end{equation*}

\vspace{2mm}
Finally, these results derive in the following theorem, which states the maximum integration error in terms of $\mathcal{P}$.\\

\begin{Th}
    Let $\mathcal{H}$ be a reproducing kernel Hilbert space with reproducing kernel $K$ such that condition \textbf{$C$} is satisfied, and let $\mathcal{P} = \{ \textbf{x}_0,...,\textbf{x}_{N-1} \} \subset [0,1)^d$ be a point set with $N$ elements. Then
    \begin{equation} \label{worst-case error original}
        e^2(\mathcal{H},\mathcal{P}) = \int_{[0,1]^d} \int_{[0,1]^d} K(\textbf{x},\textbf{y}) \,\mathrm{d}\textbf{x} \,\mathrm{d}\textbf{y} - \frac{2}{N} \sum_{n=0}^{N-1} \int_{[0,1]^d} K(\textbf{x}_n,\textbf{y}) \,\mathrm{d}\textbf{y} + \frac{1}{N^2} \sum_{n,m=0}^{N-1} K(\textbf{x}_n,\textbf{x}_m)
    \end{equation}
\end{Th}

\vspace{4mm}
On the other hand, there is the problem of choosing the nodes that will be used for the QMC rule. One way of solving this problem is to use what's called a \textit{lattice point set}.\\

\begin{Def}
    Let $d,N \in \mathbb{N}$, $N\geq 2$ and let $\textbf{g}\in \mathbb{Z}^d$. A point set $\mathcal{P}(\textbf{g},N) = \{ \textbf{x}_0,...,\textbf{x}_{N-1} \}$ with 
    \begin{equation*}
        \textbf{x}_n := \Big\{ \frac{n}{N} \textbf{g} \Big\} \hspace{2mm} \forall n=0,1,...,N-1
    \end{equation*}
    is called a lattice point set and $\textbf{g}$ is its \textit{generating vector}.
\end{Def}

\vspace{2mm}
These lattice point sets have the following property,
\begin{equation*}
    \sum_{n=0}^{N-1} \exp{2\pi i n\textbf{h} \cdot \textbf{g} / N} = \Bigg\{ 
    \begin{matrix}
        N & \text{if } \textbf{h} \cdot \textbf{g} \equiv 0 \hspace{2mm} (\text{mod } N)\\
        0 & \text{if } \textbf{h} \cdot \textbf{g} \not\equiv 0 \hspace{2mm} (\text{mod } N),
    \end{matrix}
\end{equation*}
which allow them to get low values for the worst-case error (see eq. \ref{worst-case error original}). In order to do so, it is enough to find a generating vector $\textbf{g} \in \{ 0,...,N-1 \}^d$. Since checking $N^d$ integer vectors is not feasible in practice for big $N$ and $d$ values, an algorithm was developed that makes it able to create the generating vector component-by-component (CBC).\\

\begin{Algorithm} \label{CBC}
     \textbf{(CBC Algorithm).} Let $d,N \in \mathbb{N}$.
     \begin{enumerate}
         \item Choose $g_1 = 1$.
         \item For $k=2,3,...,d$, choose $g_k \in \{ 1,...,N-1 \}$ that minimizes $R_N((g_1,...,g_{k-1},z))$ as a function of $z \in \{ 1,...,N-1 \}$, 
     \end{enumerate}
\end{Algorithm}
where
\begin{equation*}
    R_N(\textbf{g}) := \sum_{\textbf{h}\in C_d^*(N)\cap \mathcal{L(\textbf{g},N)}} \frac{1}{r_1(\textbf{h})},
\end{equation*}
\begin{equation*}
    r_1(\textbf{h}) = \prod_{i=1}^d r_1(h_i),
\end{equation*}
\begin{equation*}
    r_1(h) := \max{\{1,\abs{h}\}},
\end{equation*}
\begin{equation*}
    C_d^*(N) := \left[ \left( -N/2,N/2 \right] \cap \mathbb{Z} \right]^d\setminus\{0\}
\end{equation*}
and
\begin{equation*}
    \mathcal{L}(\textbf{g},N) := \{ \textbf{h} \in \mathbb{Z}^d : \textbf{h} \cdot \textbf{g} \equiv 0 \hspace{2mm} (\text{mod } N) \}
\end{equation*}
is the \textit{dual lattice} of the lattice point set $\mathcal{P}(\textbf{g},N)$ (\cite{Leobacher_Pillichshammer___2013___Introduction_to_Quasi_Montecarlo_Methods.pdf}).\\

Having said that, several QMC constructions exist for which the optimal rate of convergence $\mathcal{O}(N^{-1}\log(N)^{1/2})$ yields asymptotically, but it's yet unknown what the optimal point set is (in the sense that it's a global minimizer of the worst case integration error). However, for small Fibonacci numbers $N$, it is known that the unique minimizer of the QMC worst-case error in the reproducing kernel Hilbert space $H_{mix}^1$ of 1-periodic functions with mixed smoothness is the Fibonacci lattice. What's more, for $N=1,2,3,5,7,8,12,13$ the optimal point sets are integration lattices (\cite{Hinrichs.pdf}).\\





%Explicar point-sets (los de mi TFG y los del nuevo paper)
\subsection{Optimal Point Sets for Quasi-Monte Carlo (QMC) Integration}
%Básicamente hablar de los puntos de Hinrichs

Let's consider the reproducing kernel Hilbert space $H_{mix}^1$ of 1-periodic functions with mixed smoothness. It's reproducing kernel is a tensor product kernel that reads
\begin{equation*}
    K_{d,\gamma}(\textbf{x},\textbf{y}) := \prod_{i=1}^{d} K_{1,\gamma}(x_i,y_i),
\end{equation*}
where $\textbf{x}=(x_1,...,x_d), \textbf{y}=(y_1,...,y_d)\in [0,1)^d$, 
\begin{equation*}
    K_{1,\gamma}(x,y) = 1 + \gamma k(\abs{x-y}),
\end{equation*}
\begin{equation*}
    k(t) = \frac{1}{2} (t^2 - t + \frac{1}{6})
\end{equation*}
and $\gamma > 0$ is a parameter. \\

As a result, if we consider a point set $\mathcal{P} = \{ \textbf{x}_0,...,\textbf{x}_{N-1} \}$, then minimizing the worst-case error $e(H_{mix}^1,\mathcal{P})$ with respect to the Hilbert space norm corresponding to the reproducing kernel $K_{d,\gamma}$ is equivalent to minimizing the double sum
\begin{equation*}
    G_{\gamma}(\textbf{x}_0,...,\textbf{x}_{N-1}) = \sum_{n,m=0}^{N-1} K_{d,\gamma}(\textbf{x}_n,\textbf{x}_m).
\end{equation*}

It is known that the Fibonacci lattice yields the optimal point configuration for integrating periodic functions, which may suggest that the optimal point configurations are integration lattices or at least lattice point sets, however, that's not always true. Nonetheless, integration lattices are always local minima of $e(H_{mix}^1,\mathcal{P})$. What's more, for small $\gamma$ values the optimal points are always close to a lattice point set, which for $d=2$ gives point sets of the form
\begin{equation*}
    \Bigg\{ \left(\frac{i}{N},\frac{\sigma(i)}{N}\right) : i=0,...,N-1 \Bigg\},
\end{equation*}
where $\sigma$ is a permutation of $\{ 0,1,...,N-1 \}$.\\


In order to obtain these optimal point sets various considerations have to be contemplated. First, lets consider univariate 1-periodic functions $f : \mathbb{R} \longrightarrow \mathbb{R}$ given by their values on the torus $\mathbb{T} = [0,1)$.\\

\begin{Def}
    Let $k\in \mathbb{Z}$, the \textit{$k$-th Fourier coefficient} of a function $f\in L_2(\mathbb{T})$ is defined as:
    \begin{equation*}
        \widehat{f}_k = \int_0^1 f(x) \cdot \exp{2\pi ikx} \,\mathrm{d}x
    \end{equation*}
\end{Def}

\vspace{2mm}
In the univariate \textit{Sobolev space} $H^1(\mathbb{T}) = W^{1,2}(\mathbb{T})\subset L_2(\mathbb{T})$ of functions with first weak derivatives bounded in $L_2$, the following definition gives a Hilbert space norm on $H^1(\mathbb{T})$ depending on $\gamma>0$,
\begin{equation*}
    \norm{f}_{H^{1,\gamma}(\mathbb{T})}^2 = \widehat{f}_0^2 + \gamma \cdot \sum_{k\in \mathbb{Z}} \abs{2\pi k}^2 \widehat{f}_k^2 = \left( \int_{\mathbb{T}} f(x) \,\mathrm{d}x \right)^2 + \gamma \cdot \int_{\mathbb{T}} f'(x)^2 \,\mathrm{d}x,
\end{equation*}
where $f \in H^1(\mathbb{T})$ is a function.\\

Moreover, the corresponding inner product is defined as
\begin{equation*}
    \langle f,g \rangle_{H^{1,\gamma}(\mathbb{T})} = \left( \int_0^1 f(x) \,\mathrm{d}x \right) \cdot \left( \int_0^1 g(x) \,\mathrm{d}x \right) + \gamma \cdot \int_0^1 f'(x)\cdot g'(x) \,\mathrm{d}x 
\end{equation*}
where $H^{1,\gamma}(\mathbb{T})$ denotes the Hilbert space $H^1(\mathbb{T})$ equipped with this inner product.\\

What's more, $H^{1,\gamma}(\mathbb{T})$ is continuously embedded in $C^0(\mathbb{T})$, thus, $H^{1,\gamma}(\mathbb{T})$ is a reproducing kernel Hilbert space whose reproducing kernel $K_{1,\gamma} : \mathbb{T}\times\mathbb{T} \longrightarrow \mathbb{R}$ is \textit{positive definite} and reads
\begin{equation*}
    K_{1,\gamma}(x,y) := 1 + \gamma \cdot \sum_{k\in \mathbb{Z}\setminus\{0\}} \abs{2\pi k}^{-2} \cdot \exp{2\pi ik(x-y)} = 1 + \gamma \cdot k(\abs{(x-y)}), 
\end{equation*}
where $k(t) = \frac{1}{2} (t^2 - t + \frac{1}{6})$.\\

Also, this kernel has the following property,
\begin{equation*}
    f(x) = \langle f(\cdot),K(\cdot,x) \rangle_{H^{1,\gamma}(\mathbb{T})}, \hspace{2mm} \forall f\in H^1(\mathbb{T}).
\end{equation*}

That being said, the tensor product space $H_{mix}^{1,\gamma}(\mathbb{T}^2) := H^1(\mathbb{T})\otimes H^1(\mathbb{T}) \subset C(\mathbb{T}^2)$ has a reproducing kernel given by
\begin{equation*}
    K_{2,\gamma}(\textbf{x},\textbf{y}) = K_{1,\gamma}(x_1,y_1) \cdot K_{1,\gamma}(x_2,y_2) = 1 + \gamma \cdot k(\abs{x_1-y_1}) + \gamma \cdot k(\abs{x_2-y_2}) + \gamma^2 \cdot k(\abs{x_1-y_1})\cdot k(\abs{x_2-y_2})
\end{equation*}

\vspace{2mm}
Furthermore, the squared worst-case error of a QMC rule based on a point set $\mathcal{P}=\{\textbf{x}_0,...,\textbf{x}_{N-1}\}$ in $H_{mix}^{1,\gamma}(\mathbb{T}^2)$ is
\begin{equation*}
    e(H_{mix}^{1,\gamma}(\mathbb{T}^2),\mathcal{P})^2 = -1 + \frac{1}{N^2} \sum_{n=0}^{N-1} \sum_{m=0}^{N-1} K_{2,\gamma}(\textbf{x}_n,\textbf{x}_m).
\end{equation*}

\vspace{2mm}
In order to acquire an optimal point set, we have to minimize the worst-case error.\\

\textbf{Note}: So as to not confuse elements $\textbf{x}_n \in \mathcal{P}$ with their coordinates $\textbf{x}_n = (x_{n,1},x_{n,2})$ and simplify the notation, a given point $\textbf{x}_n \in \mathcal{P}$ will have coordinates $\textbf{x}_n = (x_n,y_n)$, that is, the first coordinate of a point in $\mathcal{P}$ will be denoted with an '$x$', whereby the second coordinate will be denoted with a '$y$'.\\

For the squared worst-case error we have
\begin{multline*}
    e(H_{mix}^{1,\gamma}(\mathbb{T}^2),\mathcal{P})^2 = -1 + \frac{1}{N^2} \sum_{n,m=0}^{N-1} K_{2,\gamma}(\textbf{x}_n,\textbf{x}_m) = -1 + \frac{1}{N^2} \sum_{n,m=0}^{N-1} K_{1,\gamma}(x_n,x_m) K_{1,\gamma}(y_n,y_m) \\ 
    = -1 + \frac{1}{N^2} \sum_{n,m=0}^{N-1} (1 + \gamma \cdot k(\abs{x_n-x_m}) + \gamma \cdot k(\abs{y_n-y_m}) + \gamma^2 \cdot k(\abs{x_n-x_m})\cdot k(\abs{y_n-y_m})) \\
    =\frac{\gamma}{N^2} \cdot \sum_{n,m=0}^{N-1} (k(\abs{x_n-x_m}) +  k(\abs{y_n-y_m}) + \gamma \cdot k(\abs{x_n-x_m})\cdot k(\abs{y_n-y_m})) \\
    = \frac{\gamma \cdot (2\cdot k(0)+ \gamma \cdot k(0)^2)}{N} + \frac{2\gamma}{N^2} \cdot \sum_{n=0}^{N-2} \sum_{m=n+1}^{N-1} (k(\abs{x_n-x_m}) +  k(\abs{y_n-y_m}) + \gamma \cdot k(\abs{x_n-x_m})\cdot k(\abs{y_n-y_m}))
\end{multline*}

\vspace{2mm}
Therefore, minimizing the worst-case error is equivalent to minimizing either
\begin{equation*}
    F_{\gamma}(\textbf{x},\textbf{y}) := \sum_{n=0}^{N-2} \sum_{m=n+1}^{N-1} (k(\abs{x_n-x_m}) +  k(\abs{y_n-y_m}) + \gamma \cdot k(\abs{x_n-x_m})\cdot k(\abs{y_n-y_m}))
\end{equation*}
or
\begin{equation*}
    G_{\gamma}(\textbf{x},\textbf{y}) := \sum_{n,m=0}^{N-1} (1 + \gamma \cdot k(\abs{x_n-x_m})) \cdot (1 + \gamma \cdot k(\abs{y_n-y_m})).
\end{equation*}
$G_{\gamma}$ is sometimes used for theoretical considerations, however, $F_{\gamma}$ has less summands, thus it's better for numerical implementation.\\

Now, let $\tau,\sigma \in S_N$ be two permutations of $\{ 0,1,...,N-1 \}$ and define the set
\begin{equation*}
    D_{\tau,\sigma} = \{ \textbf{x}\in [0,1)^N, \textbf{y}\in [0,1)^N : x_{\tau (0)} \leq x_{\tau (1)} \leq ... \leq x_{\tau (N-1)}, \hspace{1mm} y_{\sigma (0)} \leq y_{\sigma (1)} \leq ... \leq y_{\sigma (N-1)} \}
\end{equation*}
on which $\abs{x_n - x_m} = s_{n,m}(x_n-x_m)$ holds for $s_{n,m} \in \{ -1,1\}$. It turns out that the restriction of $F_{\gamma}$ to $D_{\tau,\sigma}$, namely $F_{\gamma}(\textbf{x},\textbf{y})_{\mid D_{\tau,\sigma}}$, is a convex polynomial of degree 4 in $(\textbf{x},\textbf{y})$ for sufficiently small $\gamma$.\\

\begin{Prop}
    $F_{\gamma}(\textbf{x},\textbf{y})_{\mid D_{\tau,\sigma}}$ and $G_{\gamma}(\textbf{x},\textbf{y})_{\mid D_{\tau,\sigma}}$ are convex if $\gamma \in [0,6]$.
\end{Prop}

\begin{proof} Because of how $F_{\gamma}(\textbf{x},\textbf{y})_{\mid D_{\tau,\sigma}}$ and $G_{\gamma}(\textbf{x},\textbf{y})_{\mid D_{\tau,\sigma}}$ have been defined, we only need to prove the claim above for one of them. Lets prove it for $G_{\gamma}(\textbf{x},\textbf{y})_{\mid D_{\tau,\sigma}}$.\\

The sum of convex functions is convex and $f(x-y)$ is convex if $f$ is convex, thus, we have to prove that $f(r,s) = (1+\gamma\cdot k(r))\cdot (1+\gamma\cdot k(s))$ is convex for $r,s\in [0,1]$. To prove it, we will see if the Hesse matrix $\mathcal{H}(f)$ is positive definite for $0\leq\gamma<6$. We have $k'(r) = (2r-1)/2$ and $k''(r) = 1$, thus $f_{rr}(r,s) = \gamma \cdot (1+\gamma\cdot k(s))$ is positive if $\gamma<24$ because $\min_{s\in[0,1]}{k(s)} = -1/24$. Therefore, it's enough to check if $\mathcal{H}(f)$ is positive. Since $f_{rs} = \gamma^2 \cdot (2r-1)/2 \cdot (2s-1)/2$, we have
\begin{multline*}
    \mid \mathcal{H}(f) \mid = (1+\gamma\cdot k(r))\cdot(1+\gamma\cdot k(s)) - \gamma^2 \cdot \left( r-\frac{1}{2} \right)^2 \cdot \left( s-\frac{1}{2} \right)^2 > 0 \\
    \Longleftrightarrow (1+\gamma\cdot k(r))\cdot(1+\gamma\cdot k(s)) > \gamma^2 \cdot \left( r-\frac{1}{2} \right)^2 \cdot \left( s-\frac{1}{2} \right)^2.
\end{multline*}
Hence we can check if
\begin{equation*}
    1+\gamma\cdot k(r) = 1 + \frac{\gamma}{2} \cdot \left( r^2-r+\frac{1}{6} \right) > \gamma \cdot \left( r - \frac{1}{2} \right)^2,
\end{equation*}
which is the case for $0\leq\gamma\leq 6$ and $r\in [0,1]$. For $\gamma=6$ extra arguments are necessary, so we omit this case.
\end{proof}

\vspace{2mm}
What's more, since
\begin{equation*}
    [0,1)^N\times[0,1)^N = \cup_{(\tau,\sigma) \in S_N\times S_N} D_{\tau,\sigma},
\end{equation*}
the global minimum of $F_{\gamma}$ on $[0,1)^N\times[0,1)^N$ can be obtained by computing
\begin{equation*}
    \arg \min_{(\textbf{x},\textbf{y})\in D_{\tau,\sigma}}{F_{\gamma}(\textbf{x},\textbf{y})}
\end{equation*}
for all $(\tau,\sigma) \in S_N\times S_N$ and choosing the smallest of all the local minima.\\

Also, symmetries of the torus $\mathbb{T}^2$ allow to reduce the number of regions $D_{\tau,\sigma}$ for which the optimization problem has to be solved. However, these symmetries don't change the worst-case error for the considered classes of periodic functions. Since implementing all of the torus symmetries is difficult, only some of them have been considered. Nevertheless, using the torus symmetries it can always be arranged that $\tau = id$ and $\sigma(0) = 0$ together with fixing the point $\textbf{x}_0 = (x_0,y_0) = (0,0)$ leads to the sets
\begin{equation*}
    D_{\sigma} = \{ \textbf{x}\in [0,1)^N, \textbf{y}\in [0,1)^N : 0 = x_0 \leq x_1 \leq ... \leq x_{N-1}, \hspace{1mm} 0 = y_0 \leq y_{\sigma (1)} \leq ... \leq y_{\sigma (N-1)} \},
\end{equation*}
where $\sigma \in S_{N-1}$ denotes a permutation of $\{1,2,...,N-1\}$.\\

Lets define the symmetrized metric 
\begin{equation*}
    d(n,m) := \min{ \{\abs{n-m}, N-\abs{n-m} \}} \hspace{2mm} \text{ for } 0\leq n,m \leq N-1.
\end{equation*}

\vspace{2mm}
\begin{Def}
    We define the \textit{set of semi-canonical permutations} as a set $\mathcal{C}_N \subset S_N$ whose permutations $\sigma$ satisfy
    \begin{enumerate}
        \item $\sigma(0) = 0$
        \item $d(\sigma(1),\sigma(2)) \leq d(0,\sigma(N-1))$
        \item $\sigma(1) = \min{ \{ d(\sigma(n),\sigma(n+1)) \mid n=0,1,...,N-1 \} }$
        \item $\sigma$ is lexicographically smaller than $\sigma^{-1}$
    \end{enumerate}
    $\sigma(N)$ is identified with $0=\sigma(0)$.
\end{Def}

\vspace{2mm}
As a result, we get the following lemma.\\

\begin{Lemma}
    For any $\sigma \in S_N$ such that $\sigma(0)=0$, there exists a semi-canonical permutation $\sigma'$ such that $D_{\sigma}$ and $D_{\sigma'}$ are equivalent (up to torus symmetry).
\end{Lemma}
\vspace{2mm}

Thus, only semi-canonical $\sigma$ need to be considered.\\

Moreover, considering the objective function only in domains $D_{\sigma}$ makes it strictly convex if $\textbf{x}_0 = (x_0,y_0) = (0,0)$.\\

\begin{Prop}
    $F_{\gamma}(\textbf{x},\textbf{y})_{\mid D_{\sigma}}$ and $G_{\gamma}(\textbf{x},\textbf{y})_{\mid D_{\sigma}}$ are strictly convex if $\gamma \in [0,6]$.
\end{Prop}

\begin{proof} Similarly to the last Proposition, it's enough to prove the result for $G_{\gamma}(\textbf{x},\textbf{y})_{\mid D_{\sigma}}$.\\

The sum of a convex and a strictly convex function is strictly convex,  thus we only need to check if 
\begin{multline*}
    f(x_1,...,x_{N-1},y_1,...,y_{N-1}) = \sum_{n=1}^{N-1} (1 + \gamma \cdot k(\abs{x_n-x_0}))(1 + \gamma \cdot k(\abs{y_n-y_0})) \\
    = \sum_{n=1}^{N-1} (1 + \gamma \cdot k(x_n))(1 + \gamma \cdot k(y_n))
\end{multline*}
is a strictly convex function on $[0,1]^{N-1}\times[0,1]^{N-1}$. In the last Proposition we actually proved that $f_n(x_n,y_n) = (1 + \gamma \cdot k(x_n))(1 + \gamma \cdot k(y_n))$ is strictly convex if $(x_n,y_n)\in [0,1]^2$ for a fixed $n$. Therefore, the strict convexity of $f$ follows from the following lemma.
\end{proof}

\vspace{2mm}
\begin{Lemma}
    Let $D_i \in \mathbb{R}^{d_i}$ be convex domains and let $f_i : D_i \longrightarrow \mathbb{R}, \hspace{1mm} i=1,...,p$ be strictly convex functions. Then
    \begin{equation*}
        \begin{matrix}
            f: & D=D_1\times ... \times D_p & \longrightarrow & \mathbb{R} \\
               & (z_1,...,z_p) & \longmapsto & \sum_{i=1}^{p} f_i(z_i)
        \end{matrix}
    \end{equation*}
    is strictly convex.
\end{Lemma}
\vspace{2mm}
Whereby each $D_\sigma$ has a unique point where the minimum of $F_\gamma$ is attained.\\


That being said, we want to find the smallest of all the local minima of $F_{\gamma}$ on each region $D_{\sigma} \subset [0,1)^N\times [0,1)^N$ for all semi-canonical permutations $\sigma \in \mathcal{C}_N \subset S_N$ to determine the global minimum. That gives for each $\sigma \in \mathcal{C}_N$ the following constrained optimization problem,
\begin{equation*}
    \min_{(\textbf{x},\textbf{y}) \in D_{\sigma}}{F_\gamma (\textbf{x},\textbf{y})} \hspace{2mm} \text{subject to } v_n(\textbf{x}) \geq 0 \text{ and } w_n(\textbf{y}) \geq 0, \hspace{2mm} \forall n=1,...,N-1,
\end{equation*}
where 
\begin{equation*}
    v_n(\textbf{x}) = x_n-x_{n-1} \text{ and } w_n(\textbf{y}) = y_{\sigma(n)} - y_{\sigma(n-1)} \hspace{2mm} \text{for } n=1,...,N-1.
\end{equation*}

What's more, we know that the necessary and sufficient (due to local strict convexity) conditions for $(\textbf{x},\textbf{y})\in D_{\sigma}$ to be local minima are
\begin{equation*}
    \frac{\partial}{\partial x_k}F_{\gamma}(\textbf{x},\textbf{y}) = 0 \hspace{2mm} \forall k=1,...,N-1
\end{equation*}
and
\begin{equation*}
    \frac{\partial}{\partial y_k}F_{\gamma}(\textbf{x},\textbf{y}) = 0 \hspace{2mm} \forall k=1,...,N-1,
\end{equation*}
where the partial derivatives of $F_{\gamma}$ are given by the formulas from the following proposition.\\

\begin{Prop}
    Given a permutation $\sigma \in \mathcal{C}_N$, the partial derivative of $F_{\gamma \mid D_{\sigma}}$ with respect to the second component $\textbf{y}$ is given by
    \begin{equation*}
        \frac{\partial}{\partial y_k}F_{\gamma}(\textbf{x},\textbf{y})_{\mid D_{\sigma}} = y_k \left( \sum_{n=0, n\neq k}^{N-1} c_{n,k} \right) - \sum_{n=0, n\neq k}^{N-1} c_{n,k} \cdot y_n + \frac{1}{2}\cdot \left( \sum_{n=0}^{k-1} c_{n,k} \cdot s_{n,k} - \sum_{m=k+1}^{N-1} c_{k,m} \cdot s_{k,m} \right),
    \end{equation*}
    where $s_{n,m} := \text{sign}(y_n-y_m)$ and $c_{n,m} := 1 + \gamma \cdot k(\abs{x_n-x_m}) = c_{m,n}$.
    The analogue for the partial derivatives with respect to $\textbf{x}$ is obtained by interchanging $\textbf{x}$ with $\textbf{y}$ and using $c_{n,m} = 1 + \gamma \cdot k(\abs{y_n-y_m})$ and $s_{n,m} = -1$.
    The second order derivatives with respect to $\textbf{y}$ are given by
    \begin{equation*}
        \frac{\partial^2}{\partial y_k \partial y_j}F_{\gamma}(\textbf{x},\textbf{y})_{\mid D_{\sigma}} = \Bigg\{ 
        \begin{matrix}
            \sum_{n=0}^{k-1} c_{n,k} + \sum_{n=k+1}^{N-1} c_{n,k} & \text{for } j=k \\
            -c_{k,j} & \text{for } j\neq k
        \end{matrix}
        , \hspace{2mm} k,j \in \{ 1,...,N-1 \}.
    \end{equation*}
    The analogue for $\frac{\partial^2}{\partial x_k \partial x_j}F_{\gamma}(\textbf{x},\textbf{y})_{\mid D_{\sigma}}$ is obtained by interchanging $\textbf{x}$ with $\textbf{y}$ and using $c_{n,m} = 1 + \gamma \cdot k(\abs{y_n-y_m})$.
\end{Prop}
\begin{proof} The partial derivative of $F_{\gamma \mid D_{\sigma}}$ with respect to $\textbf{y}$ is
\begin{multline*}
    \frac{\partial}{\partial y_k}F_{\gamma}(\textbf{x},\textbf{y})_{\mid D_{\sigma}} = \sum_{n=0}^{N-2} \sum_{m=n+1}^{N-1} \frac{\partial}{\partial y_k} k(\abs{y_n-y_m})(1+\gamma \cdot k(\abs{x_n-x_m})) + \frac{\partial}{\partial y_k} k(\abs{x_n-x_m}) \\
    = \sum_{n=0}^{N-2} \sum_{m=n+1}^{N-1} \frac{\partial}{\partial y_k} k(\abs{y_n-y_m})c_{n,m} + \frac{\partial}{\partial y_k} k(\abs{x_n-x_m}) = \sum_{n=0}^{N-2} \sum_{m=n+1}^{N-1} c_{n,m} \frac{\partial}{\partial y_k} k(\abs{y_n-y_m}) \\
    = \sum_{n=0}^{N-2} \sum_{m=n+1}^{N-1} c_{n,m} k'(s_{n,m}(y_n-y_m)) \cdot \Bigg\{ 
    \begin{matrix}  
        s_{n,m} & \text{for } n=k \\
        -s_{n,m} & \text{for } m=k \\
        0 & \text{else}
    \end{matrix} \\
    = \sum_{m=k+1}^{N-1} c_{k,m} s_{k,m}\left( s_{k,m}(y_k-y_m) - \frac{1}{2} \right) - \sum_{n=0}^{k-1} c_{n,k} s_{n,k} \left( s_{n,k}(y_n-y_k) - \frac{1}{2} \right) \\
    = y_k \left( \sum_{n=0, n\neq k}^{N-1} c_{n,k} \right) - \sum_{n=0, n\neq k}^{N-1} c_{n,k} y_n + \frac{1}{2} \left( \sum_{n=0}^{k+1} c_{n,k} s_{n,k} - \sum_{m=k+1}^{N-1} c_{k,m} s_{k,m} \right).
\end{multline*}
From this expression the second derivative of $F_{\gamma \mid D_{\sigma}}$ with respect to $\textbf{y}$ is trivially obtained. The derivatives for $\textbf{x}$ are obtained analogously.
\end{proof}
\vspace{2mm}

Now we can approximate local minima of $F_{\gamma}$ on a given $D_{\sigma}$. Doing this for all $\sigma \in \mathcal{C}$ we can obtain a candidate for the global minimum, however, due to the finite precision of floating point arithmetic we can't be sure to be close to the actual global minimum. Luckily, it is possible to compute a lower bound for the optimal point set for each $D_{\sigma}$. This is because a lower bound on a function $F$ can be obtained for convex problems with linear inequality constraints using the \textit{Lagrangian}
\begin{multline*}
    \mathcal{L}_F(\textbf{x},\textbf{y},\bm{\lambda},\bm{\mu}) := F(\textbf{x},\textbf{y}) - \bm{\lambda}^T \cdot \textbf{v}(\textbf{x}) - \bm{\mu}^T \cdot \textbf{w}(\textbf{y}) = F(\textbf{x},\textbf{y}) - \sum_{n=1}^{N-1} (\lambda_n \cdot v_n(\textbf{x}) + \mu_n \cdot w_n(\textbf{y})).
\end{multline*}
As a result, we get
\begin{equation*}
    \min_{(\textbf{x},\textbf{y})\in D_{\sigma}}{F(\textbf{x},\textbf{y})} \geq \mathcal{L}_F(\Tilde{\textbf{x}},\Tilde{\textbf{y}},\bm{\lambda},\bm{\mu})
\end{equation*}
for all $(\Tilde{\textbf{x}},\Tilde{\textbf{y}},\bm{\lambda},\bm{\mu})$ that satisfy
\begin{equation*}
    \nabla_{(\textbf{x},\textbf{y})}\mathcal{L}_F (\Tilde{\textbf{x}},\Tilde{\textbf{y}},\bm{\lambda},\bm{\mu}) = 0 \hspace{2mm} \text{with } \bm{\lambda},\bm{\mu}\geq 0,
\end{equation*}
where $\nabla_{(\textbf{x},\textbf{y})} = (\nabla_{\textbf{x}},\nabla_{\textbf{y}})$ and $\nabla_{\textbf{x}}$ denotes the gradient of a functions with respect to the variables in $\textbf{x}$.\\

Therefore, we have to find an admissible point $(\Tilde{\textbf{x}},\Tilde{\textbf{y}},\bm{\lambda},\bm{\mu})$ for each $D_{\sigma}$ which yields a lower bound that is larger than the candidate for the global minimum. The following theorem is also relevant.\\

\begin{Th}
    For $\sigma\in \mathcal{C}_N$ and $\delta > 0$ let $(\Tilde{\textbf{x}}_{\sigma},\Tilde{\textbf{y}}_{\sigma}) \in D_{\sigma}$ be a point such that
    \begin{equation} \label{uno}
        \frac{\partial}{\partial x_k}F(\Tilde{\textbf{x}}_{\sigma},\Tilde{\textbf{y}}_{\sigma}) = \delta \hspace{2mm} \forall k = 1,...,N-1
    \end{equation}
    and
    \begin{equation} \label{dos}
        \frac{\partial}{\partial y_k}F(\Tilde{\textbf{x}}_{\sigma},\Tilde{\textbf{y}}_{\sigma}) = \delta \hspace{2mm} \forall k = 1,...,N-1.
    \end{equation}
    Then the following inequality holds,
    \begin{multline} \label{Wolf}
        F(\textbf{x},\textbf{y}) \geq F(\Tilde{\textbf{x}}_{\sigma},\Tilde{\textbf{y}}_{\sigma}) - \delta \cdot \sum_{n=1}^{N-1} ((N-n)\cdot v_n(\Tilde{\textbf{x}}_{\sigma}) + \sigma (N-n) \cdot w_n(\Tilde{\textbf{y}}_{\sigma})) \\
        > F(\Tilde{\textbf{x}}_{\sigma},\Tilde{\textbf{y}}_{\sigma}) - \delta \cdot N^2 \hspace{2mm} \forall (\textbf{x},\textbf{y})\in D_{\sigma}.
    \end{multline}
\end{Th}

\begin{proof} Let $\textbf{P}_{\sigma} \in \{ -1,0,1 \}^{(N-1)\times(N-1)}$ be the permutation matrix associated to $\sigma \in S_{N-1}$ and let
\begin{equation*}
    \textbf{B} := 
    \begin{pmatrix}
        1&-1&0&...&0&0\\
        0&1&-1&...&0&0\\
        \vdots& & &\ddots& &\vdots\\
        0& &...&0&1&-1\\
        0& &...& &0&1
    \end{pmatrix} \in \mathbb{R}^{(N-1)\times(N-1)}.
\end{equation*}
The partial derivatives of $\mathcal{L}_F$ with respect to $\textbf{x}$ and $\textbf{y}$ are given by
\begin{equation*}
    \nabla_{\textbf{x}}\mathcal{L}_F(\textbf{x},\textbf{y},\bm{\lambda},\bm{\mu}) = \nabla_{\textbf{x}}F(\textbf{x},\textbf{y}) - 
    \begin{pmatrix}
        \lambda_1 - \lambda_2\\
        \vdots\\
        \lambda_{N-2} - \lambda_{N-1}\\
        \lambda_{N-1}
    \end{pmatrix}
    = \nabla_{\textbf{x}}F(\textbf{x},\textbf{y}) - \textbf{B}\bm{\lambda}
\end{equation*}
and
\begin{equation*}
    \nabla_{\textbf{y}}\mathcal{L}_F(\textbf{x},\textbf{y},\bm{\lambda},\bm{\mu}) = \nabla_{\textbf{y}}F(\textbf{x},\textbf{y}) - 
    \begin{pmatrix}
        \mu_{\sigma(1)} - \mu_{\sigma(2)}\\
        \vdots\\
        \mu_{\sigma(N-2)} - \mu_{\sigma(N-1)}\\
        \mu_{\sigma(N-1)}
    \end{pmatrix}
    = \nabla_{\textbf{y}}F(\textbf{x},\textbf{y}) - \textbf{B}\textbf{P}_{\sigma}\bm{\mu}.
\end{equation*}

Therefore, by choosing
\begin{equation*}
    \bm{\lambda} = \textbf{B}^{-1} \nabla_{\textbf{x}}F(\Tilde{\textbf{x}}_{\sigma},\Tilde{\textbf{y}}_{\sigma}) \hspace{2mm} \text{ and } \hspace{2mm} \bm{\mu} = \textbf{P}_{\sigma}^{-1} \textbf{B}^{-1} \nabla_{\textbf{y}}F(\Tilde{\textbf{x}}_{\sigma},\Tilde{\textbf{y}}_{\sigma})
\end{equation*}
we get
\begin{equation*}
    \nabla_{\textbf{x}}F(\Tilde{\textbf{x}},\Tilde{\textbf{y}}) = \textbf{B} \bm{\lambda} \hspace{2mm} \text{ and } \hspace{2mm} \nabla_{\textbf{y}}F(\Tilde{\textbf{x}},\Tilde{\textbf{y}}) = \textbf{B}\textbf{P}_{\sigma}\bm{\mu}.
\end{equation*}
Since 
\begin{equation*}
    \textbf{B}^{-1} := 
    \begin{pmatrix}
        1&1&...&1\\
        0&1&...&1\\
        \vdots&0&\ddots&\vdots\\
        0&...&0&1
    \end{pmatrix} \in \mathbb{R}^{(N-1)\times(N-1)},
\end{equation*}
it turns out that $\textbf{y},\bm{\lambda} > 0$, which together with \textit{Wolfe duality} gives the first inequality of equation \eqref{Wolf}. In order to get the second inequality it has to be noted that $0 \leq \abs{v_n(\textbf{x})},\abs{w_n(\textbf{y})} \leq 1$ and that $2\cdot \sum_{n=1}^{N-1} \sigma(N-n) = 2 \cdot \sum_{n=1}^{N-1} n = (N-1)(N-2) < N^2$.
\end{proof}
\vspace{2mm}

Having said that, suppose that $(\textbf{x}^*,\textbf{y}^*)\in D_{\sigma^*}$ is a candidate for an optimal point set. If we can find points $(\Tilde{\textbf{x}}_{\sigma},\Tilde{\textbf{y}}_{\sigma}) \in D_{\sigma}$ that fulfill equations \eqref{uno} and \eqref{dos} for all other $\sigma \in \mathcal{C}_N$ and
\begin{equation*}
    F(\Tilde{\textbf{x}}_{\sigma},\Tilde{\textbf{y}}_{\sigma}) - \delta N^2 \geq F_{\gamma}(\textbf{x}^*,\textbf{y}^*) \hspace{2mm} \text{for } \delta > 0,
\end{equation*}
we can be sure that $D_{\sigma^*}$ is the unique domain $D_{\sigma}$ that contains the globally optimal point set.\\

In conclusion, to find the global minimum $(\textbf{x}^*,\textbf{y}^*)$ of $F_{\gamma}$, one has to first compute
\begin{equation*}
    \sigma^* := \arg\min_{\sigma \in \mathcal{C}_N}{\min_{(\textbf{x},\textbf{y})\in D_{\sigma}}{F_{\gamma}(\textbf{x},\textbf{y})}},
\end{equation*}
in order to obtain a candidate point set $(\textbf{x}^*,\textbf{y}^*)\in D_{\sigma^*}$ for the global minimum of $F_{\gamma}$. After that, one has to compute lower bounds for all the other domains $D_{\sigma}$ with $\sigma\in \mathcal{C}_N$. That way, if we obtain for each $\sigma$ a point $(\Tilde{\textbf{x}}_{\sigma},\Tilde{\textbf{y}}_{\sigma})$ such that
\begin{equation*}
    \min_{(\textbf{x},\textbf{y})\in D_{\sigma^*}}{F_{\gamma}(\textbf{x},\textbf{y})} \approx \theta_N := F_{\gamma}(\textbf{x}^*,\textbf{y}^*) < \mathcal{L}_F(\Tilde{\textbf{x}}_{\sigma},\Tilde{\textbf{y}}_{\sigma}) - 2N^2\delta \leq F_{\gamma}(\textbf{x},\textbf{y}),
\end{equation*}
we could be sure that the global optimum is indeed located in $D_{\sigma^*}$ and that $(\textbf{x}^*,\textbf{y}^*)$ is a good approximation to it (\cite{Hinrichs.pdf}).\\


The optimal point sets for a parameter value $\gamma=1$ obtained by \cite{Hinrichs.pdf} using this method are shown in Figure \ref{fig:Hinrichs}. In this work we will use those point sets so as to test the efficiency of the new method.\\

%What's more, new research regarding optimal point sets has been published. This time, \cite{NewPointSets} constructed optimal $L_{\infty}$ star discrepancy sets. As a result, we will also be testing how good these new optimal point sets are for our method.\\

%
%
%
%
%
%
%
%
%
%
%
%
%
%
%
%
%
%
%
%
%
%
%
%
%
%
%
%
%
%
%
%
%\subsection{Optimal $L_{\infty}$ Star Discrepancy Sets}






%Funcionamiento de lo que hará el programa.
\subsection{Functioning of our Method}
%OBS: NUESTRO TEST SYSTEM DE QUADRATS ESTA FORMADO POR QUADRATS EN LOS
%PUNTOS DE LOS OPTIMAL POINT SETS

Our method combines Stereology with AI using Quasi-Monte Carlo Integration and Optimal Point Sets in the process. CLIP-EBC was trained to be used with an entire image. Instead, our method consists of applying the CLIP-EBC model to image crops of a given size whose left corners are located at each point contained in the chosen Optimal Point Set. Since this approach resembles the working of a test system of quadrats, we then use Equation \ref{Conteo_usada} to estimate the total count values, substituting the expectation (using Equation \ref{QMC}) by a weighted sum of the estimated count values for the image crops.\\ 

Just to clarify how the image crops were made, let $N$ be the total number of points contained in the chosen Optimal Point Set. For each point, we crop the image in such a way that the exact location of the point in the image coincides with the left corner of the image crop. Then, for an image with height $H$ and width $W$, the width of the crop is obtained as $\text{\textit{width}} = W/N$ and the height is obtained as $\text{\textit{height}} = H/N$. Note that the corner positions of an image crop may not be integers as a result of the latter calculations, thus, the resulting corner positions where rounded to the lower integer in order for the corner positions to coincide with image pixels. Also, note that points in \cite{Hinrichs.pdf} are normalized to $[0,1]^2$, thus, for an image with height $H$ and width $W$ the points are scaled accordingly to $[0,W]\times [0,H]$.
%% DomingoMaster: Added
This reduces the rounding errors and allows to work most of the case with integers, because $W,H$ are typically power of two.
\\








\chapter{Experimental results}\label{cap:Resultados}
\section{Linear Feedback Shift Register (LFSR)}
El \textit{Linear Feedback Shift Register} es un registro de desplazamiento en el que la entrada se calcula a través de una función lineal. Uno de sus usos es cifrar una secuencia binaria (a traves de la suma de un \textit{keystream} a los datos en claro), o para generar una secuencia aleatoria de dígitos a partir de una semilla o estado inicial. A la función que describe el calculo de la siguiente entrada del registro también se la conoce como polinomio de \textit{feedback}.  %no estoy seguro de si necesitabas la secuencia inicial para el crack% \\\\
Para ser más precisos teniendo una cadena binaria $b$.
\[ b = b_0, b_1, ... , b_n\]
Siendo \(n\) el numero de bits del estado inicial del LFSR, se aplicará sobre esta cadena un xor sobre ciertas posiciones de tal manera que nos de como output el bit en la posicion 0, del nuevo estado de la cadena/secuencia. A las posiciones sobre las que se aplica el xor se les llama según la literatura científica \textit{taps}.

Así que teniendo la secuencia $b$. Y por ejemplo, \textit{taps} en las posiciones 0, 1 y 4. Siendo \(b>=4\). El siguiente estado de la secuencia se computará de esta manera
\[ output = b_0 \oplus b_1 \oplus b_4 \]
\[ b = output, b_0, b_1, b_2, b_3, b_4, ..., b_{n-1}\]

Con imágenes se ve mucho más claro la naturaleza del LFSR como proceso recursivo. Pongamos como ejemplo la secuencia ``0100``, y pongamos como \textit{taps} la posición 0 y la posición 1. \\ 
\begin{figure}[h] 
    \centering
    \includegraphics[width=0.5\textwidth,keepaspectratio]{img/lfsr_01.png} .
    \label{fig:mi_imagen}
\end{figure}
\\
En la figura, se aprecia como el XOR entre la posición 0 de la cadena binaria, y la posición 1, resulta en el bit que ocupará la posición 3 de la cadena. Todos los bits del estado anterior se desplazan una posición hacia la derecha eliminando el bit en posicion 0 del estado anterior. Y por último se repite, la operación XOR.
\newpage
\begin{figure}[h] 
    \centering
    \includegraphics[width=0.5\textwidth,keepaspectratio]{img/lfsr02.png} .
    \label{fig:mi_imagen}
\end{figure}
\noindent El polinomio del LFSR del ejemplo, es un polinomio primmitivo ya que para que el registro vuelva a tener el valor de la semilla es necesario hacer $2^n-1$ iteraciones, ese menos uno se debe a que evidentemente el registro no puede valer 0 ya que en ese caso sin importar el número de iteraciones siempre permanecería en 0.\\\\
Para concluir esta sección y esclarecer del todo este tema de los registro de desplazamiento, en la siguiente figura tenéis un pseudocódigo naive para implementar cualquier LFSR.\\ 
\begin{algorithm}
\caption{Algoritmo lfsr(s, t, k)}\label{alg:zero}
\KwData{$s$ = secuencia binaria, $t$ = lista con las taps positions, k = numero de steps/iteraciones}
\KwResult{s después de k iteraciones}
\For{$i = 0; i < k; i++$}{
    $output \gets s[taps[0]]$\;
    \For{$j = 0; j < len(t); j++$} {
        $output \gets output \oplus s[taps[j]]$\;
    }
    $s = output, s[len(s)-1]$\;
}
\end{algorithm}
\\
Cabe destacar, lo increíblemente simple que es, para lo complejos que pueden llegar a ser los output a los que da salida este algoritmo. Aunque existen técnicas para decodificar estas secuencias y obtener tanto el polinomio como el estado inicial. %La implementación es extremadamente rápida en hardware ya que las operaciones necesarias para generar las secuencias, son dos, desplazamientos y xor. (poner arriba con alguna que otra frase)% \\

\subsection{Fibonacci y Galois}
Hay dos tipos principales de registros de desplazamiento lineal. El que hemos explicado hasta ahora se le denomina LFSR de Fibonacci, aunque este no fuera el autor se le llama así por su relación con las recurrencias. Este tipo de LFSR puede ser de período máximo, es decir si está constituido por un polinomio primitivo, o puede describirlo otro polinomio que no sea primitivo, y entonces tener un período menos a $2^n - 1$. \\\\
El otro tipo es más díficil de explicar y cómo en las siguientes secciones no tiene especial impacto, tan sólo lo mencionaré, se trata del LFSR de Galois (sí, el del duelo y las cartas), en este registro de desplazamiento lineal algunos de los bits se desplazan y otros son el resultado de operaciones entre las posiciones del propio registro.
\\




\section{Algoritmo de Berlekamp-Massey}
Antes de pasar al algoritmo sobre el que se central de traducción de una secuencia binaria a su FSR mínimo. Debemos explicar el procedimiento que toma como referencia, este es el algoritmo de \textit{Berlekamp Massey}. El algoritmo original, Berlekamp, del que deriva Berlekamp Massey, fue creado para decodificar códigos BCH (Bose-Chaudhuri-Hocquenghem), estos códigos se utilizan para codificar información de manera redundante por si hubiera algún fallo en la transmisión o en el almacenado de los datos poder corregirlos a través de Forward Error Correction (FEC). \\\\
El algoritmo de Berlekamp-Massey, también llamado por su creador James Massey, \textit{LFSR Synthesis Algorithm (Berlekamp iterative algorithm)} en el articulo original \cite{massey1969shift}, es capaz de encontrar la recurrencia lineal más corta para cualquier secuencia numérica finita. Esto tiene mucho que ver con la modificación de este algoritmo que utilizaremos en la siguiente sección para calcular la complejidad no lineal de un secuencia binaria. Además también es el culpable de que el cifrado a través de LFSR sea inseguro, ya que si conoces una parte del texto claro de tamaño suficiente puedes recuperar el \textit{keystream} con el que se cifro.\\\\
Primero explicaré en que consiste el algoritmo para encontrar la recurrencia lineal mas corta para cualquier secuencia de números en base decimal. Dada una secuencia $s = \{s_0, s_1, \dots\}$ diremos que existe una secuencia de recurrencia lineal $c = \{c1, c2, \dots, c_n\}$, las cuales cumplen.
\[s_i = \Sigma_{j=1}^n c_js_{i-j} \quad \texttt{para $i \geq n$}\]
Para $i \geq n$ tan solo quiere decir que es para todos los números fuera de los casos base ya que los casos base de la recurrencia lineal serán desde $s_0$ hasta $s_n$. Un ejemplo de recurrencia lineal muy típico es la sucesión de Fibonacci o su primo hermana la sucesión de Lucas. \\\\
Berlekamp Massey se basa en lo siguiente. Teniendo una secuencia $s$ cualquiera y un conjunto de coeficientes vacío, ir rellenando el conjunto de coeficientes según se recorre la secuencia $s$ y en caso de que ocurra un fallo, aplicar una función correctora $d$. También hay que guardar copias de c y tener un índice de un fallo pasado, en el ejemplo veremos para que y por qué.\\\\ 
Veamos esto con un ejemplo sencillo, dada la secuencia $s = \{ 1, 3, 9, 15, 9, -81 \}$ y el conjunto $c = \{ \ \}$, asumiremos que si c está vacío, c será igual a 0. Definiremos como $pc$ (\textit{past c}) al conjunto que nos servirá como copia de c, el cual comienza vacío como $c$. Y la variable donde guardaremos el índice al fallo pasado como $f$.
\\\\
En la primera iteración: \\
\(i=0 \quad s_0 = 1 \\
c(0) = 0 \neq s_0\\
\)\\
Como da igual a que valor pongamos coeficiente añadamos a $c$ ya que $s_0$ ahora será caso base, en mi caso añado a $c$ tantos ceros como iteraciones hayan pasado hasta el primer fallo en este caso la primera así que $c = \{ 0 \}$.
\\\\
En la segunda iteración: \\
\(i=0 \quad s_1 = 3 \\
c(1) = c_1 * s_{1-1} = 0 \neq s_1\\
\)\\
Al ser el segundo fallo, ya hay que aplicar la función correctora $d$, el cálculo de esta siempre es igual. Primero hacer que $d$ sea igual a $-1 * pc$
Aquí está el pseudocódigo, podeis encontrar una implementación en Python escrita por mí en \href{https://github.com/domingoUnican/TFGPedroCastro/blob/main/code/code_proofs/simple_berlekamp_massey.py}{simplebm.py} basado en uno de los códigos de \href{https://mzhang2021.github.io/cp-blog/berlekamp-massey/}{artículo berlekamp massey}.



\begin{algorithm}
\caption{Algoritmo Berlekamp-Massey}\label{alg:two}
\KwData{$s$ secuencia binaria}
\KwResult{El LFSR mínimo}
$c \gets [ \ ]$\;
$oldc \gets [ \ ]$\;
$f \gets -1$\;
\For{$i = 0; i < len(s); i++$}{
  \eIf{$f == -1$}{
     $c \gets resize(c, i+1)$\;
     $f \gets i$\Comment*[r]{Para la primera vez}
  }{
  $d \gets - oldc$\;
  $d.insert(0,1)$\;
  $df1 \gets \sum_{j=1}^{len(d) + 1}d_{j-1}*s[f+1-j]$\;
  $coefficient = \frac{delta}{df1}$\;
  $d \gets d*coefficient$\;
  $d \gets \texttt{addzerostoleft(i-f-1, d)}$\Comment*[r]{suma i-f-1 ceros a la izq}
  $copyc \gets c$\;
  \If{\(len(c) > len(d)\)}
  {
   $c \gets \texttt{addzerostoright(len(d) - len(c), d)}$
  }
  
  }
}
\end{algorithm}



NOTA: Añadir nota breve de aplicaciones del algoritmo de berlekamp massey


\begin{comment}
Seccion por pulir hasta aclarar el sentido de la diferencia simetrica de conjuntos en este algo
\State $h \gets( \{ a - b + it for it in g \}$
\State $f \gets f \symdif h$
\end{comment}
%\EndIf
%\EndFor
%\EndProcedure
%\end{algorithmic}
%\end{algorithm}
\newpage
\section{Non lineal Feedback Shift Register (NLFSR / FSR)}
NOTA: Puede que sea importante explicar que es, y la conexion con el Lempel Ziv Compression)\\\\
Uno de los objetivos de este proyecto ha sido intentar simplificar la función $h$ resultante del algoritmo del articulo \cite{limniotis2007nonlinear}. El paso previo ha sido analizar y dividir por partes el algoritmo hasta entenderlo completamente, esto ha sido de las cosas que más tiempo me han llevado. \\\\
En esta sección, se tratará de explicar este algoritmo recursivo como a mi me hubiera gustado que me lo explicasen. Este procedimiento toma como base Berlekamp-Massey, el cual resuelve eficientemente el cálculo del LFSR mínimo, y lo adapta para conseguir un registro de desplazamiento no lineal (NLFSR) con un 
polinomio mínimo para una secuencia binaria dada. Se dice que un registro de desplazamiento es no lineal, si en su polinomio se utiliza tanto la operación XOR (suma) como la operación AND (multiplicacion).\\\\
Antes de describir el algoritmo, debo hacer un breve resumen sobre la notación que utilizaré durante esta sección para describir conceptos como una secuencia numérica cualquiera. La notación está basada en la del artículo original. Pero creo que es necesario, plasmarlo aquí, para una mejor lectura de esta parte del trabajo. Para definir un \textit{minterm} usaremos la notación, $\underline{x_b} = x_1^{b1} ,..., x_n^{bn}$, donde $b = (b_1,...,b_n) \in \mathbb{F}_2^{n}$, siendo $x_i^0 = x'_i$ y $x_i^1 = x_i$, por ejemplo, el \textit{minterm} de la secuencia 011 es $x'_1x_2x_3$. Para las cadenas o secuencias, usaré la siguiente notación, diremos que una secuencia binaria de $N$ elementos se escribe $y^{N}$. Si queremos describir un segmento de esa secuencia, escribiremos $y_{i}^{j}$ siendo $i\leq j \leq N-1$. Esto es así porque trataremos las posiciones de las secuencias como si fuesen \textit{arrays} o listas en cualquier lenguaje de programación, es decir, siendo la posición inicial el 0. Definimos la complejidad no lineal de una cadena como $c(y^{N})$. He de decir también que las siglas FSR se utilizaran para referirse al NLFSR (\textit{Non Lineal Feedback Shift Register}). \\\\%(NOTA: puede que queden más cosas, por ahora lo dejo así...)% 
\newpage
Ahora desglosaremos paso por paso el algoritmo y señalaremos las similitudes con Berlekamp-Massey.

\begin{algorithm}
\caption{Algoritmo FSR mínimo}\label{alg:three}
\KwData{$s$ = cadena binaria}
\KwResult{El non linear FSR mínimo}
$k \gets 0$\;
$m \gets 0$\;
$h \gets y_0$\;
\For{$i \gets 1,...,N-1$} {
    $d \gets y_i - h(y_{i-1},\dots,y_{i-m})$\;
    \If{$d \neq 0$} {
    \uIf{$ m = 0 $} {
        $k \gets i$\;
        $m \gets i$\;
    }
    \ElseIf{$k \leq 0$} {
        $t \gets \texttt{Eigenvalue$(y^{i})$}$\;
        \If{$t < i + 1 - m$} {
            $k \gets i + 1 - t - m$\;
            $m \gets i + 1 - t$\;
        }
    }
    $f \gets (x_{1} + y'_{i+1})...(x_m+y'_{i-m})$\;
    $h \gets h + f$\;
    }
    $k \gets k - 1$\;
}
\end{algorithm}
\newpage
Empezaremos definiendo algunas de las variables y su significado. La variable $k$ está denominada salto y es la distancia entre $c(y^{n})$ y $c(y^{n-1})$. Uno de los teoremas del artículo prueba [X, pp 2] que (falta explicar un poco el por que de esta formula y su relación con $k(y^n)$)
\[k = n - m - (t^{(n-1)}_0 + T^{(n - 1)})\]
Este teorema se usará en el algoritmo para el cálculo del salto. Siendo $t_0^{(n - 1)}$, el preperiodo de la secuencia $y^{n-1}$ y $T^{(n - 1)}$ el periodo de $y^{n-1}$. Es importante aclarar, que $t_0$ y $T$ son el preperiodo y el periodo en el caso de que sean los números enteros más pequeños que cumplan.

\[y_i = y_{i+T} \ \text{para todo} \ i \geq t_0, \text{siendo} \  t_0 \geq 0, T > 0\]
Esto realmente es una forma para los no matemáticos muy enrevesada de decir, que $T$ es la longitud del periodo y $t_0$ el índice en el cual la periodicidad comienza, y a estas dos variables se les llaman \textit{preperiodo} y \textit{periodo}. \\\\
%\href{http://google.com}{Google} ejemplo vinculo

Todo esto será importante, ya que en el caso en el que el salto sea menor o igual que 0, esto podría hacer que la complejidad no lineal de la secuencia cambiara.\\\\
Por otro lado, la variable $m$ es la complejidad no lineal de la secuencia a evaluar. Y h es la función de feedback que acabara generando la secuencia original a partir de un segmento inicial de esa propia secuencia, el núcleo del algoritmo es entender como y por qué se actualiza esta función.\\\\
Pues bien, la clave de todo está en la variable $d$ que se recalcula en cada iteración y que llamaremos \textit{discrepancia}. El cálculo de la \textit{discrepancia} en la iteración $i$ consiste en computar si el resultado de la función de feedback ($h$) de esa iteración difiere de la posición $i$ en la secuencia. En caso de que difiera, se sumará el minterm que hace que esa posición se calcule como corresponde a la función de feedback, actualizando esta última. Cabe destacar, que no porque haya una \textit{discrepancia} tiene que aumentar la complejidad no lineal de la secuencia.\\\\
Existen varios casos si existe una \textit{discrepancia}, el primero de ellos es que aun no se haya inicializado la complejidad no lineal de $y^{n}$, es decir, si m es igual a 0. En ese caso se les asigna tanto al salto, como a $m$ el valor de $n$, es decir la posición de la primera \textit{discrepancia}. Todo esto parece muy complicado pero con el ejemplo al final de la sección, se entenderá mucho mejor. El segundo caso, en el que $m$ no es 0 y $k \leq 0$, se calculará el \textit{eigenvalue} de la secuencia desde la posicion 0 hasta la posicion $n$. Al leer las definiciones de  \textit{eigenvalue} y \textit{eigenwords} en [x], no termine de entender a que se referían, así que fui a la fuente original el artículo \cite{lempel1976complexity}. Esto me ayudo a aclarar los conceptos de vocabulario, prefijos \textit{propios}, \textit{eigenvalues} y \textit{eigenwords} de una cadena $y^n$, junto con otros conceptos como historial, historial exhaustivo, reproducibilidad y producibilidad. \\\\
Vamos a definir brevemente cada una de estas cosas, para entender bien, como y para qué necesita el algoritmo este \textit{eigenvalue}. El vocabulario de una secuencia $y^{N}$ son los subconjuntos de esa misma secuencia formado por todas las \textit{subcadenas} $y_i^{j}$ siendo $0 < i < j$ y $j < N$. Por ejemplo, de esta secuencia 01001, su vocabulario sería.
\[v(y^{N}) = \{0, 1, 01, 10, 00, 01, 010, 100, 001, 0100, 1001, 01001 \}\]
El siguiente paso es encontrar las \textit{eigenwords} (\textit{palabras propias}). Una palabra del vocabulario, es una palabra \textit{propia} si no pertenece a ninguno de los vocabularios de los prefijos propios (\textit{proper} prefixes) de la cadena original ($y^{N}$). Un prefijo \textit{propio} es un segmento desde $i$ hasta $j$, donde $0 < i < j < N-2$, es decir que la longitud del prefijo propio $Q$ debe ser menor que la longitud de la cadena original $y^{N}$, $l(Q) < l(y^{N})$. El conjunto de palabras \textit{propias} se le denomina vocabulario \textit{propio} ($e$)(\textit{eigenvocabulary}). El vocabulario propio del ejemplo anterior sería.
\[e(y^{N}) = \{ 001, 1001, 01001 \}\]
Y ahora ya tenemos la forma de calcular el \textit{eigenvalue} $k(y^N)$, pues el cardinal del vocabulario \textit{propio} es igual al \textit{eigenvalue}, $|e(y^N)| = k(y^N)$, en este caso 3. Toda este proceso tan largo, no es necesario para el calculo del \textit{eigenvalue}, hay una manera mucho más sencilla y óptima, la desarrollaremos un poco más adelante. Podéis consultar una implementación \textit{naive} de este proceso en el programa \href{https://github.com/domingoUnican/TFGPedroCastro/blob/main/code/code_proofs/naive_eigenvalue.py}{naiveEigenvalue.py}.\\\\ Aunque pueda parecerlo, todo este pretexto no es arbitrario, pues con esta explicación tenemos la capacidad de entender, el por qué de la condición en el pseudocódigo, $t < n + 1 - m$. En caso de que la condición, se evalúe como cierta, esto quiere decir que siendo $m = c(y^{n-1})$, $k(y^n) < n - m$. Luego la subcadena de tamaño $m$, $y^{n-2}_{n-m-1}$ se repite dos veces con diferentes sucesores en $y^{n-1}$, esto provoca que se tenga que sumar un \textit{minterm} con una variable más al polinomio, o dicho de otra manera, corregir la función de \textit{feedback}, línea del pseudocódigo \textit{20}. Sin embargo, si la condición se evalúa como falsa, $k(y^{n-1}) >= n - m$, la subcadena $y^{n-2}_{n-m-1}$ ahora es única por lo que a pesar de que se tenga que añadir un nuevo \textit{minterm} a la función de \textit{feedback} el numero de parámetros del polinomio no aumentará, por lo que $c(y^n)$ tampoco. (nota: pordría desarrollar esto un poco mas).\\\\
Hay un punto que hemos pasado por alto, se trata de la función de la variable $k$ (\textit{salto}) en el pseudocódigo. La función principal de esta variable en el algoritmo es ahorrarnos el cálculo del \textit{eigenvalue} en momentos en los que no es necesario calcularlo ya que sabemos que no han pasado suficientes iteraciones como para que pueda haber otro \textbf{salto} en la complejidad no lineal.\\\\
Esta variable solamente condiciona el flujo del programa en el \textit{if} $k <= 0$, y como se puede observar se resta 1 a esta variable por iteración, y cuando es menor o igual a 0, se le asigna la diferencia entre $c(y^n)$ y $c(y^{n-1})$. Esto en realidad, se hace para que tengan que pasar $c(y^n) - c(y^{n-1})$ iteraciones, hasta que pueda darse un nuevo salto en la complejidad ya que a partir de esas iteraciones es cuando \textbf{podrían} aparecer 2 subcadenas de longitud $m$ con diferentes sucesores, es en ese momento donde el cálculo del \textit{eigenvalue} cobra relevancia.\\\\
Con todo este conocimiento previo, vamos a analizar un ejemplo en el que iremos explicando paso a paso, que hace el algoritmo. Tomando la secuencia binaria $y^{10} = 1101010011$, antes de entrar al bucle $k$ y $m$ serán 0. Y $h(x) = y_0 = 1$, téngase en cuenta que al contador/iterador lo llamaremos $i$ por comodidad. \\\\
En la primera iteración del bucle: \\
\(i=1 \quad y_1 = 1 \\
\text{Calculo de la discrepancia de $y^1 = 11$:}\\
\text{$h(1) = 1$, en este momento, h(x) es una función constante}\\
\text{¿es $h(1)$ igual a $y_1$?, sí, por lo que no hay discrepancia}\\
\text{$k = k - 1$, y seguimos adelante}
\)\\\\
En la segunda iteración del bucle: \\
\(i=2 \quad y_2 = 0 \\
\text{Calculo de la discrepancia de $y^2 = 110$:}\\
\text{$h(1) = 1$, h(x) sigue siendo una función constante}\\
\text{¿es $h(1)$ igual a $y_2$?, No, así que tenemos nuestra primera \textbf{discrepancia}}\\
\text{Al ser la primera discrepancia, $m = 0$, así que}\\
m = i  \quad k = i \\
\text{\textbf{Importante}, toca corregir nuestra función \textit{fsr}, añadiendo el \textit{minterm}} \\
\underline{x_{\{ 11 \}}} = x_0*x_1 \\
h(x) = h(x) + \underline{x_{\{ 11 \}}} = 1 + x_0*x_1 \\ 
\text{Y como en todas las iteraciones, }\\
\text{$k = k - 1$}
\)\\\\
Antes  pasar a la siguiente iteración, quería hacer un apunte, parece una tontería pero es que gracias al evaluar la función en modulo 2. Es muy simple e intuitivo que al añadir el \textit{minterm} de esa posición, la función se corrija. Ya que si antes te daba 1, con el \textit{minterm} (pues el minterm equivale a 1, en esa posicion de la cadena) cambiará a 0, y viceversa.\\\\
En la tercera iteración del bucle: \\
\(i=3 \quad y_3 = 1 \\
\text{Calculo de la discrepancia de $y^3 = 1101$:}\\
\text{La subcadena que servirá como input hay que darle la vuelta para que coinci- }\\
\text{da con el orden de las variables.}\\
y^{3-1}_{3-m} = y^2_{1} = 10 \\
\texttt{reverse($y^2_{1}$) = 01} \\
h(x_0=0, x_1=1) = 1+0*1 = 1\\
\text{¿es $h(01)$ igual a $y_3$?, Sí, no se realizan cambios en $h(x)$}\\
\text{$k = k - 1$}                      
\)\\\\
\\\\
En la cuarta iteración ocurre una \textit{discrepancia} que no cambia la complejidad no lineal $m = c(y^n)$, tan solo añade a la función $h$ el \textit{minterm}, $x_0*x'_1$. En la quinta y sexta no hay discrepancias. Sin embargo, en la séptima iteración ocurre lo siguiente\\\\
\(i=7 \quad y_7 = 0 \\
\text{Calculo de la discrepancia de $y^7 = 11010100$:}\\
y^{7-1}_{7-m} = y^6_{5} = 10 \\
\texttt{reverse($y^6_{5}$) = 01} \\
h(x_0=0, x_1=1) = 1+0*1+0*(1+1) = 1\\
\text{¿es $h(01)$ igual a $y_7$?, No}\\
\)\\
Ya que hay más de una subcadena $y^{i-1}_{i-m}$ con distintos sucesores, hay que aumentar la complejidad no lineal $m$, en concreto $y^6_{5}$ se repite 2 veces previamente pero con el mismo sucesor. En esta iteración es la tercera vez que se repite la subcadena pero se da una discrepancia. 
\[
{\text{\large $10 \quad \textcolor{green}{\xrightarrow{}} \quad 10 \quad \textcolor{green}{\xrightarrow{}} \quad 10 \quad \textcolor{red}{\xrightarrow{}} \quad 0$}}
\]
Al darse este caso, el \textit{eigenvalue} será menor que $n + 1 - m$. Por lo que se actualizará el $k$ y $m$. \\
\(
k = n + 1 - t - m \\
m = n + 1 - t \\
\text{El siguiente paso es sumar el \textit{minterm} con $m$ ya actualizada, por lo que }\\
\underline{x_{\{ 01010 \}}} = x'_0*x_1*x'_2*x_3*x'_4 \\
h(x) = h(x) + \underline{x_{\{ 01010 \}}} = 1 + x_0*x_1 + x_0*x'_1 + x'_0*x_1*x'_2*x_3*x'_4\\ 
\text{Y al final de la iteracion, }\\
k = k - 1                     
\)\\\\
(Falta una pequeña explicacion de por k = n + 1 - t -m y m = n + 1 - t)\\
En la octava iteración no se dan discrepancias. Y por último, en la novena se da una discrepancia que no aumenta $m$, sumando el minterm $x_0*x'_1*x'_2*x_3*x'_4$. Aquí termina el ejemplo, habiendo recorrido toda la cadena, y quedando una función final de feedback tal que así 
\[h(x) = 1 + x_0*x_1 + x_0*x'_1 + x'_0*x_1*x'_2*x_3*x'_4 + x_0*x'_1*x'_2*x_3*x'_4 \]
Con esta función podríamos a partir de los primeros 5 digitos de la secuencia recuperar la secuencia entera, aplicando la función de feedback, añadiendo el resultado a la derecha y recalculando. Este proceso debe repetirse, siendo $w$, el numero de parámetros de la función de feedback y $r$, la longitud original de la cadena. Se aplicará el proceso $r - w$ veces, para conseguir la secuencia original. Podéis consultar también el decodificador, que he escrito en Python en \href{https://github.com/domingoUnican/TFGPedroCastro/blob/main/code/common/utils.py}{verificamfsr.py}. 

\subsection{Como encontrar el \textit{eigenvalue} de manera óptima}
Esta métrica es esencial en nuestro algoritmo previo, y ya hemos descrito anteriormente como calcularla de la manera más sencilla o "\textit{naive}". Sin embargo, en el articulo original \cite{limniotis2007nonlinear} en la sección que describe el procedimiento recursivo, indican que la manera más eficiente de calcular el \textit{eigenvalue}, es mediante el algoritmo de \textit{Knuth Morris Pratt}. Este algoritmo nos permite buscar con una complejidad temporal muy buena, un patrón en una cadena a la que cual llamaremos texto. \\\\
El algoritmo consta de dos partes, la primera, la llamaremos \textit{preprocesado}, la cual tiene una complejidad temporal $O(m)$, siendo $m$ la longitud del patrón. En esta sección, se calcula una tabla a la que normalmente se le llama \textit{longest proper boundary table}. Esta tabla, en términos prácticos, será una lista en la que cada posición indica el \textit{longest proper boundary} de la secuencia hasta ese carácter o dígito dependiendo del tipo de cadena a la que le apliquemos el algoritmo. Este valor, representa el tamaño de la subcadena más larga que es a la vez prefijo y sufijo de la cadena hasta el índice de esa posición del patrón. Esta tabla o lista, después nos ayudará en la segunda parte del algoritmo pues a \textit{grosso modo}, nos permitirá "recordar" que parte del patrón nos coincide en cada iteración.\\\\
En la segunda parte, recorremos el texto, ese bucle evidentemente tendrá complejidad $O(n)$, siendo $n$ la longitud del texto. Y lo que haremos, en realidad, es bastante simple, se recorre el texto y el patrón a la vez con dos índices distintos, $i$ para el texto y $j$ para el patrón, mientras los caracteres del texto y del patrón coincidan, ambos avanzan a la misma velocidad. Pero, cuando no coinciden si el patrón ha coincidido parcialmente con la parte del texto que se itera en ese momento, entonces es cuando entra la lista $lpb^m$ en acción. Ya que, se le asignará al índice el valor de $lpb_{j - 1}$, y  no aumentará $i$ en esa iteración. Retrocediendo el índice del patrón hasta $lpb_{j-1}$, lo que se consigue es ahorrar en iteraciones pues a pesar de que no haya coincido en ese carácter puede que coincida más atrás del patrón. En el caso, de que el patrón coincida completamente en el texto, el índice $j$ se le asigna $lpb_{m-1}$, puesto que la cuenta de coincidencia del patrón puede empezar en el índice $lpb_{m-1}$.\\\\
A continuación, tenéis una implementación general de este algoritmo, que devuelve si el patrón se encuentra o no en el texto, no es difícil imaginar que también puedes devolver el numero de coincidencias del patrón en el texto, que es justo lo que nos interesa para el \textit{eigenvalue}.\\\\
\newpage
\begin{algorithm}
\caption{Algoritmo KMP (Knuth Morris Pratt)}\label{alg:four}
\KwData{$s$ = cadena binaria}
\KwResult{True si encuentra el patrón False si no lo encuentra}
$n \gets len(t)$\;
$m \gets len(p)$\;
\tcc{Preprocesado, calculo de lpb}
$lpb \gets (0)_m$\;
\For{$i = 0; i < m; i++$} {
    $j \gets lpb_{i-1}$\;
    \While{$j > 0 \And p_j \neq p_i$} {
        $j \gets lpb_{j-1}$\;
        \eIf{\(p_j == p_i\)} {
            $lpb_i \gets j+1$\;
        }
        {
        $lpb_i \gets j$\;
        }
    }
}
\tcc{Busqueda de p en t}
$isFound \gets false$\;
$j \gets 0$\;
\For{$i = 0; i < n; i++$}{
    \While{\(j > 0 \And t_i \neq p_j\)} {
        $j \gets lpb_{j-1}$\;
    }
\If{\(t_i == p_j\)} {
    $j \gets j + 1$\;
}
\If{\(j == m\)} {
    $isFound \gets true$\;
    break\;
    \tcc{Si no se para despues de la primera coincidencia completa}
    \tcc{ $j \gets lpb_{j-1}$}
    }

}
\texttt{return} $isFound$\;
\end{algorithm}
La tabla de debajo de este párrafo se ha tomado de \cite{leiserson2022introduction}, un libro donde explican en detalle y con claridad todos estos algoritmos de búsqueda en cadenas. Lo destacable de esta tarde es la notoria diferencia entre \textit{Knuth Morris Pratt} y el \textit{Naive}, este \textit{Naive} recorre la cadena buscando el patrón y cada vez que un carácter del patrón no le coincide, reinicia la búsqueda desde el siguiente carácter es por esto por lo que nos da una complejidad de $O((n-m+1)m)$. 
\begin{center}
\begin{tabular}{ l c c }
 Algortimo & Tiempo de Preprocesado & Tiempo de ejecución \\ \cline{1-3}
 Naive & 0 & $O((n-m+1)m)$ \\  
 Rabin-Karp & $O(m)$ & $O((n-m+1)m)$ \\
 Automáta finito & $O(m|\Sigma|)$ & $O(n)$ \\
 Knuth Morris Pratt & $O(m)$ & $O(n)$
\end{tabular}
\end{center}

Nosotros aplicaremos este algoritmo para encontrar el \textit{eigenvalue} de la siguiente manera en la función \textit{Eigenvalue} del pseudocódigo. Guardaremos en una variable $p$ (patrón), la cadena \texttt{reverse($y_0^{i-1}$)}, y en otra variable $t$ (texto), la cadena $p_1^{i-1}$. Y teniendo una función \textit{knuthmp} implementada con \textit{Knuth Morris Pratt}, que podéis consultar en \href{https://github.com/domingoUnican/TFGPedroCastro/blob/main/code/minimal_fsr/knuthMorrisPratt.py}{kmp.py}, que lo que hace es devolver el numero de posiciones que se encuentran del patrón en el texto. Devolveremos
\[Eigenvalue(y^i) = i - knuthmp(t, s)\]
Esta diferencia puede dar como resultado un numero en el intervalo $[1, i]$ en función del numero de coincidencias del patrón.


\section{Formato de archivo nlfsr}
El formato de archivo nlfsr (Non Linear Feedback Shift Register) es un formato el cual codifica la secuencia inicial y el polinomio FSR. Teniendo como objetivo hacer que este tipo de archivo ocupe el menor espacio posible, para ello he propuesto que la estructura del formato fuera la siguiente.
\\
\begin{figure}[h] 
    \centering
    \includegraphics[width=\textwidth,keepaspectratio]{img/nlfsrformat_01.png} .
    \parbox{\linewidth}{\centering Formato .nlfsr}
    \label{fig:mi_imagen}
\end{figure}

\noindent El primer segmento del archivo es la longitud de la cadena original ($y^n$), contenido en 16 bits. Estos 16 bits limitan el tamaño de la secuencia a comprimir en este formato, a una secuencia de longitud máxima 65536. Este límite es arbitrario, pero una vez se haya leído toda la explicación del formato entenderéis que no es difícil adaptarlo para cadenas más largas.  El segundo segmento es la complejidad no lineal $m$ de $y^n$ contenida también en 16 bits (o 2 Bytes). El tercer segmento es la secuencia inicial que mide $m$ bits. Y el cuarto segmento es una lista de \textit{minterms}, aquí tenéis una representación de como se representa un \textit{minterm}.
\\
\begin{figure}[h] % El [H] asegura que la imagen esté exactamente en ese lugar en el documento.
    \centering
    \includegraphics[width=\textwidth,keepaspectratio]{img/nlfsrformat_02.png} % La imagen se ajustará al ancho del texto, manteniendo la proporción.
    \parbox{\linewidth}{\centering Lista de minterms en formato .nlfsr}
    \label{fig:mi_imagen} % Etiqueta para referenciar la imagen en el documento.
\end{figure}
\noindent Cada \textit{minterm} consta de un primer segmento en el que se indica el numero de variables de ese producto de literales lógicos, o dicho de otra manera la longitud. Cada bit del \textit{minterm} representara el estado de ese literal en el producto, si su valor es 1 eso significa que el literal se encuentra negado, por otro lado, si es 0, significará que es positivo. De esta manera, conseguimos almacenar toda la información consumiendo el mínimo numero de bytes posibles. \\\\
La longitud del propio \textit{minterm} actuando como delimitador entre \textit{minterms}, nos permite saber hasta donde tenemos que leer cuando haya que decodificar el archivo .nlfsr. Es importante aclarar, que no se pueden utilizar delimitadores fijos en este caso, como sí se en algunos protocolos de red, ya que toda secuencia podría estar dentro del minterm, y eso haría que se mezclasen los datos con las delimitaciones, lo cual provocaría errores en el momento de la lectura.
\\\\
He hecho unas gráficas para que se pueda ver la diferencia entre guardar un nlfsr de una secuencia con la misma estructura del formato .nlfsr pero usando caracteres ASCII, y utilizar el formato .nlfsr. Aquí tenéis las gráficas sin que se haya utilizado ningún algoritmo de compresión después. 
\begin{figure}[h] % El [H] asegura que la imagen esté exactamente en ese lugar en el documento.
    \centering
    \includegraphics[width=0.8\textwidth,keepaspectratio]{img/format_comparison_graph.jpeg}
    \parbox{\linewidth}{\centering Comparación tamaños formatos}
    \label{fig:mi_imagen} 
\end{figure}
%Encriptación distribuida,
% \subsection{Potenciales aplicaciones del formato nlfsr} (posible seccion)
% Encriptación minimizada distribuida,
\newpage
\section{Minimización de ESOP}
La minimización de expresiones de operaciones AND-OR cuenta con muchos algoritmos que hacen un buen trabajo, sin embargo la minimización de ESOP resulta ser mucho más compleja o al menos, no se han desarrollado algoritmos del mismo calibre. Pero primero, definamos que es todo esto.\\\\
A la forma de la función FSR, se la conoce como ESOP (\textit{Exclusive Sum Of Products}). Una formula lógica con esta forma, tan solo puede tener dos operaciones: XOR (suma) y AND (multiplicación). Minimizar una fórmula ESOP, significa encontrar una formula de menor tamaño que tenga la misma tabla de verdad, o el mismo mapa de Karnaugh; en definitiva, que sean equivalentes.Aquí tenéis un ejemplo de un ESOP y su minimización, aunque más tarde veremos la teoría que hay detrás.
\\\\
\noindent
\begin{minipage}{0.5\textwidth}
    \centering
    \begin{karnaugh-map}[4][2][1][$X_1X_0$][$X_2$]
        \minterms{1, 5, 6, 7}
        \maxterms{0, 2, 3, 4}
    \end{karnaugh-map}
\end{minipage}
\begin{minipage}{0.5\textwidth}
    \[
    f(X_0, X_1, X_2) = X_0X_1 \oplus X_1X_2 \oplus X_2
    \]
\end{minipage}%
\\
Ahora trataremos de encontrar una equivalencia entre $X_1X_2 \oplus X_2$ y un cubo formado por estos literales. Encontramos que
\\\\
\noindent
\begin{minipage}{0.5\textwidth}
    \centering
    \begin{karnaugh-map}[2][2][1][$X_1$][$X_2$]
        \minterms{1}
        \maxterms{0,2, 3}
    \end{karnaugh-map}
\end{minipage}
\begin{minipage}{0.5\textwidth}

     \begin{align*}
          f'(X_1, X_2) =  X_1X_2 \oplus X_2 \\
     z(X_1, X_2) =  X'_1X_2 \\
     f' = z
     \end{align*}
\end{minipage}%
%\newpage
Y sabiendo esto, sustituimos en la formula original y nos queda un ESOP minimizado

\[f_{min} = X_0X_1 \oplus X'_1X_2\]
\\
La \textit{Suma Exclusiva de Productos} se ha estudiado desde 1925 [Paper original], el artículo más reciente que he encontrado respecto a este tema es []. En este describen un algoritmo en el que mediante transformaciones de cubos, expresiones pseudokronecker, reescritura de términos \cite{brand1993minimization}, y estructuras de datos como BDD (Binary Decision Diagram). Nuestra aproximación al problema en este proyecto será mucho más sencilla pero haber leído todos estos artículos sobre el tema, me ha hecho entender más en profundidad lo complejo que es.
\\\\
A pesar de no hacer una implementación como la de [Heuristic Minimiza], en esta sección tomaremos prestada la teoría que elabora el artículo sobre los cubos que forman los ESOP. Definiremos como "cubo" a cada producto del ESOP. Definiremos como distancia entre cubos, el numero de variables que aparecen de distinta "forma" respecto al otro cubo. Una variable puede aparecer en un cubo de tres maneras: en positivo, negada, o no aparecer. Por ejemplo, un cubo $ab$ tiene distancia 1 respecto a un cubo $ab'$. Lo último, que creo que se debe tener en cuenta del articulo para la posible mejora de nuestro algoritmo, son dos proposiciones 
\begin{tcolorbox}
\textbf{Proposición 1:}\\
Si añades el mismo cubo dos veces a cualquier ESOP, la función no cambiará. \\
\textbf{Proposición 2:}\\
La suma de dos cubos con distancia 1 se puede representar en un solo cubo. 
\end{tcolorbox}
%\noindent En el ejemplo anterior aplicamos la segunda proposición para minimizar la función.





\subsection{Enumeración explicita para encontrar el ESOP de N variables de menor longitud}
¿Y si en lugar de encontrar el mínimo para una fórmula en concreto, encontramos el mínimo de símbolos que debe de tener una fórmula para todos los resultados posibles dada una formula (o ESOP) de n variables de entrada? Pues bien, esto es justo lo que hace nuestro pequeño algoritmo. En lo que sigue veremos paso a paso en que consiste. El primer paso que debemos entender es como conseguimos representar todas las formulas posibles de n variables con m simbolos, pudiendo ser cada símbolo, o bien, un XOR, o bien, un AND. La estructura que utilizamos para lograrlo es muy sencilla, es una lista de enteros. Pero tiene un truco, y es que el 0 representa el XOR o la suma, el 1 representa la operación AND o la multiplicación, y el 2 representa la negación. Cualquier entero $i > 2$ representa una variable de entrada $X_{i - 3}$. Con este sistema de representación en el que cualquier fórmula puede ser representada por una lista o cadena de enteros en la que cada posición tiene un entero $i$, $0 <= i < n+3$ de longitud, $m+(m+1)$, siendo $n$ el numero de variables de la fórmula y $m$, el numero de símbolos(XOR, AND) que puede tener la fórmula. \\\\
Para entender mejor el sistema, pondré los siguientes ejemplos. Teniendo una formula $X_0 * X_2 * X_3 + X_0$ y otra formula $X'_0 + X_1 * X_3 + X_4$, estas son sus representaciones con este sistema.
\begin{align*}
X_0 * X_2 * X_3 + X_0 = [0, 1, 3, 1, 5, 6, 3] \\
X'_0 + X_1 * X_3 + X_4 = [1, 0, 2, 3, 4, 0, 6, 7]
\end{align*}
El cálculo del numero de posibles resultados para una formula de n variables de entrada es
\[k = 2^{2^{n}}\]
Ahora debemos encontrar para cada posible resultado del conjunto de posibles resultados de tamaño k, una formula que satisfaga ese resultado. Para ello lo que haremos, será generar todas las combinaciones con repetición del conjunto $\{0,1, ..., n + 3 - 1\}$ de longitud $p$, definimos al conjunto de todas las combinaciones como $g^{(n+3-1)^p}$, cada elemento de $g$ será una posible combinación de tamaño $p$. Es importante decir que, $p$ determinará el numero de posibles símbolos que pueden haber en la fórmula. \\\\
\newpage
Es cierto que habrá algunas combinaciones que no tendrán sentido sintáctico en esta gramática que hemos creado. Por ejemplo $[2, 2, 2, 2, 2, 2, 2]$, no representa ninguna fórmula. Para distinguir entre las listas que tienen sentido de las que no, hemos escrito una función recursiva que crea una cadena de texto o \textit{string} en base a la lista. Aquí tenéis el pseudocódigo, también podéis leer el código real escrito en Python en \href{https://github.com/domingoUnican/TFGPedroCastro/blob/main/code/logicformula_solver/logicform_solver.py}{logicformulasolver.py}.
\begin{algorithm}
\caption{Algoritmo intListToStringFormula(f)}\label{alg:five}
\KwData{$f$ = lista de enteros}
\KwResult{Función lógica en formato string}
\If{\texttt{isNotEmpty(f)}} {
    $val \gets \texttt{f.pop()}$\tcp*{pops last int from f}
    
    \If{\(val == 0\)} {
        \texttt{return stringFormula(f) + '+' + stringFormula(f)}\;
    }
    \If{\(val == 1\)} {
        \texttt{return stringFormula(f) + '*' + stringFormula(f)}\;
    }
    \If{\(val == 2\)} {
        \texttt{return 'not' + stringFormula(f)}\;
    }
    \texttt{return 'x' + (val - 3)}\;
}
\texttt{return 'S'}\;
\end{algorithm}

\noindent El siguiente paso, es escribir la función para calcular cuantos posibles resultados se generan para un numero de variables dado y una longitud de la cadena de lista de enteros dada. Que lo único que hace es para cada cadena de enteros válida (validada con la función del pseudocódigo escrita encima de este párrafo). Prueba todas las posibles combinaciones de las variables de entrada de la función, es decir si por ejemplo, es una función de 3 variables de entrada probara las $2^3$ combinaciones. Recopilará todos los resultados y hará una lista con ellos. La intentará sumar a un set, la cual se sumará en función del hash calculado a esa lista internamente por Python.\\\\ Una vez se han agotado todas las combinaciones de listas de enteros ("formulas lógicas"), se comparará la longitud del set con el número de posibles resultados $k$; si estos coinciden es que hemos encontrado el tamaño mínimo de la lista de enteros para generar todas las formulas necesarias que coinciden con todos los resultados posibles. \\ 
\newpage
\noindent Aquí tenéis dos gráficas creadas con la librería \textit{matplotlib}, en el eje $X$ la primera fila es el tiempo que ha tardado en ejecutar el algoritmo, y la segunda fila es el tamaño de la lista de enteros. El eje $Y$ representa el numero de posibles resultados retornados por el algoritmo.

\begin{figure}[h!]
    \centering
    \begin{minipage}{0.5\textwidth}
        \centering
        \includegraphics[width=\linewidth]{img/2_variables.png}
        \parbox{\linewidth}{\centering ESOP de 2 variables de entrada}
    \end{minipage}%
    \begin{minipage}{0.5\textwidth}
        \centering
        \includegraphics[width=\linewidth]{img/3_variables.png}
        \parbox{\linewidth}{\centering ESOP de 3 variables de entrada}
    \end{minipage}
    \vspace{1em} % Espacio entre las filas de imágenes

    \begin{minipage}{0.5\textwidth}
        \centering
        \includegraphics[width=\linewidth]{img/4_variables.png}
        \parbox{\linewidth}{\centering ESOP de 4 variables de entrada}
    \end{minipage}
\end{figure}

\noindent Como vemos en las gráficas, el algoritmo converge en ambos casos es decir, encuentra todas las listas de enteros que satisfacen los $k$ posibles resultados para ESOP de $n$ variables. Sin embargo, a partir de ESOPs de 4 variables debido a la complejidad temporal del algoritmo, se tarda mucho en que el algoritmo converja y empeora conforme el número de variables aumenta, ya que la complejidad temporal de nuestro algoritmo es de $O((n+3)^p * 2^n)$. Con mi procesador, en 10 minutos consigue calcular la función hasta $p = 9$.
\\\\
Para calcular las combinaciones en el código original, hemos usado la librería \textit{itertools}, y en concreto la función \textit{product}, ya que fue una recomendación de mi tutor, Domingo. Este modulo de Python genera sus funciones, utilizando Cython para después crear módulos utilizables desde Python, pero escritos en C. Sentí cierta curiosidad, por como sería hacer un programa en C que calcule todas las combinaciones tal y como lo hace \textit{itertools}.\\\\
Así que me puse manos a la obra, y escribí dos versiones una secuencial \href{https://github.com/domingoUnican/TFGPedroCastro/blob/main/code/list_product/list_product.c}{secuencial.c}. y otra paralelizada con \textit{OpenMP} \href{https://github.com/domingoUnican/TFGPedroCastro/blob/main/code/list_product/list_product_parallel.c}{paralelo.c}. Si habéis leído el código sabréis que lo hago todo sobre un solo array, lo escribí así por temas de la localidad de los valores guardados. En mi \textit{hardware}, el código secuencial resultó ser un poco más lento que el de \textit{itertools}, sin embargo, para mi sorpresa el código paralelizado sí me dio mejores resultados, incluso superando a la función del módulo de Python. Si bien, es cierto que Python está "capado" por el GIL (Global Interpreter Locker), ya que a pesar de contar con módulos que ofrecen funciones de "multithreading" no se consigue en ningún momento una paralelización real, considero que los datos siguen siendo interesantes, aquí tenéis una gráfica.

\begin{figure}[h] % El [H] asegura que la imagen esté exactamente en ese lugar en el documento.
    \centering
    \includegraphics[width=0.9\textwidth,keepaspectratio]{img/programs_comparison_graph.jpeg}
    \parbox{\linewidth}{\centering Comparación tamaños formatos}
    \label{fig:mi_imagen} 
\end{figure}
\noindent El eje Y es el tiempo de ejecución en segundos, y el eje X tiene 2 listas de valores, la lista más cercana al eje es el cardinal del conjunto sobre el que se calculan todas las posibles combinaciones. y el tamaño de la lista es el número de debajo. Cabe destacar que para un conjunto de 10 elementos y un tamaño de lista 8, mi sistema operativo (Debian \textit{bookworm}) mataba el proceso de Python. Sin embargo, para los mismos parámetros mis programas escritos en C, tanto el secuencial como el paralelo, registraron 16.8 y 5.8 segundos respectivamente. Evidentemente con parámetros más altos tenía problemas de memoria debido a mi \textit{hardware} limitado.



\chapter{Conclusions and open problems}\label{cap:Conclusiones}
\section{Avances futuros}
En este trabajo se tocan muchos temas distintos. Cada uno de ellos puede expandirse mucho más de lo que se ha hecho aquí. Se me ocurren muchas posibilidades para expandir GoneFSR y la investigación de la minimización de ESOP (\textit{Exclusive Sum of Products}).\\\\
Para el compilador GoneFSR se podría trabajar en implementar operadores lógicos \textit{bitwise}, implementar instrucciones de control de flujo como \textit{break} o \textit{continue}. O en funcionalidades más complejas, como implementar paradigmas como OOP (\textit{Object Oriented Programming}) con sus respectivas clases, atributos y demás. Podría parecer complicado, pero creo que no lo es tanto ya que las clases no dejan de ser funciones, las cuales tienen otras funciones definidas dentro de su \textit{scope}. En un futuro, también me gustaría programar algún sistema de importaciones para poder trabajar sobre varios ficheros, teniendo en cuenta las importaciones circulares y otras casuísticas. Otra futura implementación podría ser la mejora en la administración de memoria con sistemas como la \textit{garbage collection} de Python que a partir de técnicas como el conteo de referencias, o el \textit{cycle detection} para evitar ciclos de referencias, gestionan la liberación de la memoria de las variables a las que no se hace referencia. Y por qué no, puestos a imaginar un gestor de módulos como \textit{pip} o \textit{npm}.\\\\
Por otro lado, se podría desarrollar un backend personalizado con optimizaciones de código usando vectorización con instrucciones SIMD (\textit{Single Instruction Multiple Data}), \textit{unrolling} para descomponer bucles y aumentar la paralelización del código, reordenación de instrucciones para minimizar tiempos de espera. Estas son sólo unas pocas líneas de investigación que se pueden seguir en el desarrollo de un backend de compilación.\\\\
En cuanto a la segunda parte del proyecto hay muchas cosas que he estado estudiando antes de escribir esta memoria que no han podido ser reflejadas en estas líneas ya que a pesar de que puede que tengan que ver, quedaban fuera de los límites del proyecto. Como por ejemplo conceptos como la síntesis de programación, la cual trata de generar programas que cumplan ciertos requisitos, estudiando algoritmos como la enumeración explicita e implícita, si queréis saber más podéis consultar estos artículos \href{https://people.csail.mit.edu/asolar/SynthesisCourse/Lecture1.htm}{\textit{programming synthesis}}. O novedosos avances del estado del arte de la inteligencia artificial, como los sistemas de agentes LLM (\textit{Large Language Model}) (p.ej \cite{islam2024mapcoder}) para generar programas con tan sólo una vaga descripción de lo requerido.\\\\
Donde más potencial de investigación en lo referido a las funciones de registro de desplazamiento estudiadas creo que existe es en la minimización de ESOP, destiné algún tiempo a estudiar el artículo \cite{mishchenko2001fast} y a estudiar código que encontre de su proyecto \textit{EXORCISM} para minimizar estas funciones lógicas. Me estaba llevando mucho tiempo y no conseguí llegar a nada en concreto, pero creo que con el suficiente tiempo se puede indagar en ese campo para buscar nuevas optimizaciones. Creo que otra aproximación adecuada para este problema de minimización podrían ser los SMT (\textit{Satisfacibility Modulo Theories}) como \textit{Z3}, y los SAT solvers, que hace relativamente poco consiguieron resolver el teorema de las tripletas booleanas pitagóricas en 4 años de CPU tal y como dice el matemático Terrence Tao en la \href{https://www.youtube.com/watch?v=e049IoFBnLA&t=1061s}{conferencia}.
\section{Conclusión}
Este proyecto ha llevado mucho trabajo, ha habido momentos muy frustrantes tanto en el desarrollo del compilador, como en la investigación de los algoritmos de registro de desplazamiento. Pero no me arrepiento en absoluto de haberme embarcado en esta empresa. \\\\
La implementación de las sentencias condicionales, trabajar con strings en llvm, añadir la capacidad de escribir funciones, parece poco pero son muchos problemas distintos a los que enfrentarse. Aunque es cierto, que una vez entiendes la estructura central del compilador cada extensión no es tan complicada pues al final son modificaciones de esa propia estructura.\\\\
Entender como funciona el algoritmo de la sección \ref{nlfsr algorithm} me llevo mucho tiempo, entender por qué se suma los minterms como se suman, el por qué del cálculo del \textit{eigenvalue}, conceptos como la complejidad de \textit{Lempel-Ziv} u otros de los que nunca había oído hablar. Intentar hacer un formato que sea óptimo para guardar la información de ese algoritmo. Todo esto fue como hacer un \textit{puzzle} en el que te dan las piezas pero no sabes cómo es la imagen final que deben de formar.\\\\
A modo de conclusión final, me gustaría reafirmar que este proceso aunque haya tenido sus malos ratos y sus momentos de satisfacción cuando todo cuadraba (o cuando parecía que todo cuadraba y al final no), creo que ha merecido la pena, y acabo el proyecto siendo un mejor ingeniero informático.



% Indique aquí el fichero .bib que contenga su bibliografía.
\bibliography{refs}

\appendix
\chapter{Appendix}
\begin{lstlisting}[style=pythonStyle]
program : statements
        | empty

statements :  statements statement
           |  statement

statement :  const_declaration
          |  var_declaration
          |  assign_statement
          |  print_statement
          |  if_statement
          |  while_statement
          |  return_statement

const_declaration : CONST ID = expression ;

var_declaration : VAR ID datatype ;
                | VAR ID datatype = expression ;

assign_statement : location = expression ;

print_statement : PRINT expression ;

coder_statement: CODER expression , expression ;

decoder_statement: DECODER expression , expression ;


if_statement : IF expression { statements }
             | IF expression { statements } ELSE { statements }
             
while_statement: WHILE expression { statements }

function_definition : FUNC ID LPAREN arguments RPAREN datatype { statements }

parameters: parameter
           | parameters COMMA parameter
           | expression
           | empty

function_calling: function_location LPAREN parameters RPAREN SEMI

arg_declaration: ID datatype

arguments: argument
           | arguments comma argument
           | arg_declaration
           | empty

expression : + expression
           | - expression
           | ! expression
           | expression && expression
           | expression || expression
           | expression == expression
           | expression != expression
           | expression < expression
           | expression <= expression
           | expression > expression
           | expression >= expression
           | expression + expression
           | expression - expression
           | expression * expression           
           | expression / expression
           | ( expression )
           | location
           | literal
           | function_call
               

literal : INTEGER     
        | FLOAT       
        | CHAR     
        | BOOL 

function_location: ID

location : ID
         ;

datatype : ID
         ;

empty    :
\end{lstlisting}






\end{document}



