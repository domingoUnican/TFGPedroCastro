In this final degree project, an interpreter for the Begone language, a programming language proposed by David M. Beazley, is created. The goal is for this implementation to be used in the course "Programming Languages."

The student carries out this implementation in Python, performing the following tasks:

\begin{itemize}
  \item Defines specifications of the tokens and implementation using regular expressions
  \item Defines an unambiguous context-free grammar for the Begone language
  \item Uses libraries to generate native code with the compiler
\end{itemize}

Additionally, this language has native functions for file compression, based on feedback registers, which are as efficient as possible. These functions:

\begin{itemize}
  \item Allow the compression and decompression of files in different formats (e.g., text, binary).
  \item Use feedback registers to optimize compression based on usage patterns and common data types.
  \item Provide a simple interface for users to easily integrate these functions into their programs written in Begone.

\end{itemize}
