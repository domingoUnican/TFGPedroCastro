%El problema del conteo de personas en imágenes resulta de vital importancia en tareas de seguridad pública y control de masas en eventos multitudinarios. La tecnología actual utiliza diferentes metodologías automáticas para diseñar y entrenar distintos modelos de redes neuronales. Por otro lado, existen métodos de conteo manual donde el objetivo es obtener una estimación con el menor número de muestras posibles. Los métodos manuales se basan en una ciencia matemática llamada estereología, pero su aplicación ha quedado en segundo plano por la ingente cantidad de datos que ahora permiten entrenar redes neuronales con bastante precisión. En este trabajo se plantea hacer una revisión sistemática de la literatura de los dos campos. Se tratará de combinar los métodos de estereología con las redes neuronales para comprobar si la estereología ha quedado suplantada o sus fundamentos pueden lograr avances en este campo. 



%La estereología es la ciencia del muestreo geométrico y permite estimar cantidades utilizando fórmulas que suelen incluir esperanzas. En este trabajo se han intentado mejorar los estimadores usados en estereología aproximando dichas esperanzas mediante el uso de integración QMC con $N$ puntos óptimos que minimizan el \textit{worst-case-error} para el \textit{reproducing kernel Hilber space} $H_{mix}^1$. Este proceso debería dar estimadores con poca varianza, mejorando la calidad de la cantidad estimada.\\
%Para comprobar el funcionamiento de este proceso se han seleccionado tres \textit{test systems} conocidos, el \textit{test system de quadrats}, el \textit{Buffon test system} y el \textit{Buffon-Steinhaus test system}, y se ha implementado en un código python para cada uno.\\
%Los resultados calculados por ordenador apuntan a que este proceso puede que funcione mejor o peor dependiendo de la isotropía, anisotropía o simetría que tiene el objeto de cuya cantidad se quiere estimar. %También es posible que cambiando un cierto parámetro $\gamma$ (ver \cite{Hinrichs.pdf}) se obtengan mejores resultados al usar ese proceso, pero aún así permite obtener estimadores precisos con bajo error.\\

%----------------------------------------------------------------------------------
%Stereology is the science of geometric sampling and it allows to estimate quantities using formulas that usually include expectations. Artificial Intelligence is a discipline in computer science that encompasses several different processes concerned with making machines smart. In this work we combined both Stereology and AI in order to test a new method for Crowd Counting, an AI field that revolves around the estimation of the number of individuals in images or videos. Our method's approach considers a modified test system of quadrats whose quadrats are only located at the points contained in an Optimal Point Set that minimizes the worst case error for Quasi-Monte Carlo Integration. By superimposing the system to an image and cropping that image by the quadrats, we can then use an AI model to count individuals on each crop, subsequently using Equations \ref{Conteo_usada} and \ref{QMC} to estimate the total count value of individuals in the original image. The chosen AI model to try our method was CLIP-EBC, as it is a public model and one of the best performing models in crowd-counting. The computed results for the validation metrics MAE and RMSE suggest that there is a lot or research to be made regarding our method, however, the best variant tested for our method is not far from performing in a similar level to that of the state-of-the-art methods in Crowd Counting. What's more, the computing times obtained for that variant are much lower than those obtained by one of the best performing methods, CLIP-EBC.\\

La estereología es la ciencia del muestreo geométrico y permite estimar cantidades utilizando fórmulas que suelen incluir esperanzas. La inteligencia artificial es una disciplina en ciencias de computación que engloba varios procesos relacionados con crear máquinas inteligentes. En este trabajo se han combinado la estereología y la IA de cara a probar un nuevo método de conteo de multitudes (Crowd Counting), el área centrada en estimar el número de individuos en imágenes o vídeos. El planteamiento de nuestro método considera un \textit{test system} (sistema de testeo) de quadrats modificado cuyos quadrats sólo se posicionan en los puntos contenidos en un conjunto de puntos óptimo que minimiza el \textit{worst case error} (error en el peor caso) para la integración Quasi-Monte Carlo. Superponiendo este \textit{test system} a una imagen y recortándola en los quadrats podemos usar el modelo de IA para contar individuos en cada recorte y, posteriormente, usar las Ecuaciones \ref{Conteo_usada} y \ref{QMC} para estimar la cantidad total de individuos en la imagen original. El modelo de IA escogido para probar nuestro método ha sido CLIP-EBC, dado que es un modelo público y uno de los modelos con mejores resultados en conteo de multitudes. Los resultados computados para las métricas de validación MAE (error absoluto medio) y RMSE (error cuadrático medio) muestran resultados esperanzadores respecto a nuestro método. %%DomingoMaster: Esto, aunque esta bien, me parece un poco repetitivo.
%Aunque, la mejor variante probada de nuestro método no está lejos de rendir a un nivel similar al de los métodos más recientes de conteo de multitudes. 
Lo que es más, los tiempos de cómputo obtenidos para dicha variante son mucho más bajos que los obtenidos por uno de los modelos que mejores resultados obtiene, CLIP-EBC.\\





