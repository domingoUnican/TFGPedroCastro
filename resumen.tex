En este trabajo de fin de grado se crea un intérprete para el lenguaje Begone, un lenguaje de programación propuesto por David M. Beazley. El objetivo es que esta implementación se utilice en la asignatura «Lenguajes de programación».

El alumno lleva esta implementación en Python, realizando las siguientes tareas:

\begin{itemize}
  \item Define especificaciones de los tokens e implementación mediante expresiones regulares
  \item Define una gramática libre de contexto para el lenguaje Begone, sin ambigüedad
  \item Utiliza librerías para que el compilador genere código nativo
\end{itemize}

Adicionalmente, este lenguaje tiene funciones nativas para comprimir ficheros, basado en registros de retroalimentación, que es lo más eficiente posible. Estas funciones:

\begin{itemize}
  \item Permiten la compresión y descompresión de ficheros de diferentes formatos (por ejemplo, texto, binario).
  \item Utilizan registros de retroalimentación para optimizar la compresión en función de patrones de uso y tipos de datos más comunes.
  \item Proporcionan una interfaz sencilla para que los usuarios puedan integrar fácilmente estas funciones en sus programas escritos en Begone.
\end{itemize}
